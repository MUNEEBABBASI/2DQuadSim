\documentclass[11pt]{article}
\usepackage[margin=1in]{geometry}

\usepackage{graphicx}
\usepackage{amsfonts}
\usepackage{amssymb}
\usepackage{amsmath}
\usepackage{amsthm}
\usepackage{latexsym}
\usepackage{graphicx}
\usepackage{algorithm} 
\usepackage{algorithmic} 
\usepackage{setspace}
\usepackage{subfig}

\begin{document}

\centering
Quadrotor with a Cable-Suspended Load in 1-Dimension \\
September 5, 2013

\raggedright

\begin{table} [h!]
\footnotesize
\begin{tabular}{ c c }
	$m_Q, m_L \in \mathbb{R}$ & Mass of quadrotor, load \\
	$f \in \mathbb{R}$ & Magnitude of thrust for quadrotor \\
	$l \in \mathbb{R}$ & Length of suspension cable \\
	$T \in \mathbb{R}$ & Magnitude of tension in cable \\
	${x}_Q, \mathbf{x}_L \in {R}^2$ & Position of center of mass of quadrotor, load \\
	${v}_Q, \mathbf{v}_L \in {R}^2$ & Velocity of center of mass of quadrotor, load \\
\end{tabular}
%\caption{Variables used}
\label{tab: vars}
\end{table}




%%%%%%%
\newpage
\small

\section{Equations of Motion} 
% Define the hybrid system
%%%%%
\[
    \Sigma : 
\begin{cases}
    \mathbf{\dot{x}}_1 = f_1(\mathbf{x}_1) + g_1(\mathbf{x}_1) \mathbf{u}_1, & \mathbf{x}_1 \not\in \mathcal{S}_1 \\
    \mathbf{x}_2^+ = \Delta_1(\mathbf{x}_1^-), & \mathbf{x}_1^- \in \mathcal{S}_1 \\
    \mathbf{\dot{x}}_2 = f_2(\mathbf{x}_2) + g_2(\mathbf{x}_2) \mathbf{u}_2, & \mathbf{x}_2 \not\in \mathcal{S}_2 \\
    \mathbf{x}_1^+ = \Delta_2(\mathbf{x}_2^-), & \mathbf{x}_2^- \in \mathcal{S}_2 
\end{cases}
\]

\[
 \Sigma :
\begin{cases}
%\mathbf{x}_1 &= [\mathbf{x}_L \ \ \mathbf{v}_L \ \ \phi_L \ \ \dot{\phi}_L \ \  \phi_Q \ \ \dot{\phi}_Q]^T \\
\dot{\mathbf{x}}_1 = 
\begin{bmatrix}
       \dot{x}_L \\ \dot{v}_L
\end{bmatrix}
= 
\begin{bmatrix}
       0 & 1 \\
       0 & 0 
\end{bmatrix}
\begin{bmatrix} 
	x_L \\ v_L 
\end{bmatrix}
+ 
\begin{bmatrix}
	0 \\ 
       \frac{1}{m_L+m_Q}
\end{bmatrix}
f 
+
\begin{bmatrix}
	0 \\ 
       -g
\end{bmatrix},
& 
\mathbf{x}_1 \not\in \{ \mathbf{x}_1 \ | \  T \equiv \| m_L(\dot{v}_L + g) \| = 0 \} \\
%%
\mathbf{x}_2^+ = 
\begin{bmatrix}
      {x}_L^+ \\ {v}_L^+ \\ {x}_Q^+ \\ {v}_Q^+ \\
\end{bmatrix}
= 
\begin{bmatrix}
       {x}_L^- \\ {v}_L^- \\ {x}_L^- + l \\ {v}_L^- + l \\
\end{bmatrix} , 
&  
\mathbf{x}_1^- \in \{ \mathbf{x}_1 \ | \  T = 0 \} \\
%%
\dot{\mathbf{x}}_2 = 
\begin{bmatrix}
       \dot{{x}}_L \\ \dot{{v}}_L \\ \dot{{x}}_Q \\ \dot{{v}}_Q \\
\end{bmatrix}
= 
\begin{bmatrix}
       0 & 1 & 0 & 0 \\
       0 & 0 & 0 & 0 \\
       0 & 0 & 0 & 1 \\
       0 & 0 & 0 & 0 \\
\end{bmatrix}
\begin{bmatrix}
       x_L \\ {{v}}_L \\ {{x}}_Q \\ {{v}}_Q \\
\end{bmatrix}
+ 
\begin{bmatrix}
      0 \\
      0 \\
      0 \\
      \frac{1}{m_Q}
\end{bmatrix}
f
+ 
\begin{bmatrix}
       0 \\
       -g \\
       0 \\
       -g \\
\end{bmatrix},
& 
\mathbf{x}_2 \not\in \{ \mathbf{x}_2 \ | \  \| {x}_Q - {x}_L \| = L \} \\
%%
\mathbf{x}_1^+ = 
\begin{bmatrix}
       {x}_L^+ \\ {v}_L^+ \\ 
\end{bmatrix}
= 
\begin{bmatrix}
       x_L^- \\
       v_Q^- \\
\end{bmatrix} , 
&  
\mathbf{x}_2^- \in \{ \mathbf{x}_2 \ | \  \| {x}_Q -  {x}_L \| = L \}
\end{cases}
\]





%%%%%%
\newpage
%%%
\subsection{When cable is taut}

Assume that the quadrotor is moving in the $y-z$ plane, where its only degree of freedom is in the $z$ direction. $+z$ is pointing upwards. \\
Constraint: ${x}_Q = {x}_L + l$ \\


\begin{align*}
\sum \mathbf{F} &= m \ddot{x} \\
-m_Q g - T + f &= m_Q \dot{v}_Q \\
-m_L g + T &= m_L \dot{v}_L
\end{align*}

Solving for $\dot{v}_L$ in terms of only $f$:
\begin{align*}
T &= - m_Q \dot{v}_Q - m_Q g + f \\
m_L \dot{v}_L &= -m_L g +(- m_Q \dot{v}_Q - m_Q g + f ) \\
\text{Differentiating the constraint gives: } & \\
\dot{x}_Q = v_Q &= \dot{x}_L = v_L \\
\dot{v}_Q &= \dot{v}_L \\
(m_L+m_Q) \dot{v}_L &= -(m_L+m_Q) g + f \\
\dot{v}_L &= -g +\frac{f}{m_L+m_Q}
\end{align*} 

This gives our equations of motion: 
\begin{align*}
\dot{x}_L &= v_L \\
\dot{v}_L &= -g +\frac{f}{m_L+m_Q}
\end{align*} 

Note that the system transitions to $\mathbf{x}_2$ when the tension is 0. The tension force in the cable can be explicitly from the equation of motion:
\begin{align*}
-m_L g + T &= m_L \dot{v}_L \\
T &= m_L (\dot{v}_L + g)
\end{align*}

Note that this force will always go from positive to negative at the transition. In order to exert a tension force on the load, there must be a net force upwards, as tension cannot act downwards on the load. Thus, when tension is non-zero, $\dot{v}_L > -g$ the load simply enters free fall at the point $\dot{v}_L=g$, and it's not feasible to have $\dot{v}_L < -g$. 







%%%%%
\newpage
\subsection{When cable is slack}

Assume that the quadrotor is moving in the $y-z$ plane, where its only degree of freedom is in the $z$ direction.  \\
No constraints. Load is in free fall while quad is controlled by its input force $f$. 

\mbox{} \newline
\begin{align*}
\sum \mathbf{F} &= m \ddot{x} \\
-m_Q g + f &= m_Q \dot{v}_Q \\
-m_L g &= m_L \dot{v}_L
\end{align*}

Solving for $\dot{v}_L$ and $\dot{v}_Q$ in terms of only $f$:
\begin{align*}
\dot{v}_L &= -g \\
\dot{v}_Q &= -g + \frac{f}{m_Q} \\
\end{align*} 

This makes our equations of motion: 
\begin{align*}
\dot{x}_L &= v_L \\
\dot{v}_L &= -g \\ 
\dot{x}_Q &= v_Q \\
\dot{v}_Q &= -g + \frac{f}{m_Q} \\
\end{align*} 


Note that the system transitions to $\mathbf{x}_2$ when the quad and load are at a distance $l$ apart again, with the load falling away from the quad. In other words, we transition when: 
\begin{align*}
(x_Q - x_L) - l &= 0
\end{align*}

When the load is falling away from the quad, we go from $(x_Q - x_L) < l$ to $(x_Q - x_L) = l$, or when $(x_Q - x_L)-l$ is increasing. This means that $\frac{dL(t)}{dt} = \frac{d ((x_Q - x_L)-l) }{dt} = v_Q-v_L \ge 0$, or $v_Q \ge v_L$. 






%%%%%%%
\newpage
\section{Differential Flatness} 
% Decouple equations
%%%%%
\subsection{$\mathbf{x}_1$ system:} 

Recall the equation of motion: 
%%
\begin{align*}
\dot{x}_L &= v_L \\
\dot{v}_L &= -g +\frac{f}{m_L+m_Q}
\end{align*}




Choose flat output $\mathbf{y} = [{x}_L]$  \\

\mbox{} \newline
Derive $\dot{x}_L = v_{L}$, $\ddot{x}_L = \dot{v}_{L}$ from differentiation of $x_L$ \\
We can simply find the nominal input force $f$ from the equation of motion:
\begin{align*}
f &= (m_L+m_Q) \dot{v}_L + g
\end{align*}







%%%%%%%
\subsection{$\mathbf{x}_2$ system:}

Recall the equations of motion:
%%%%%
\begin{align*}
\dot{x}_L &= v_L \\
\dot{v}_L &= -g \\ 
\dot{x}_Q &= v_Q \\
\dot{v}_Q &= -g + \frac{f}{m_Q} \\
\end{align*}


Flat output $\mathbf{y} = [{x}_Q]$  \\

\mbox{} \newline
${x}_L$ and ${v}_L$ are known from initial conditions because load is in free fall:

\begin{align*}
v_L(t) &= -gt + v_L(t_0) \\
x_L(t) &= -gt^2 + v_L(t_0)t + x_L(t_0) 
\end{align*}

Derive $\dot{x}_Q = v_{Q}$, $\ddot{x}_Q = \dot{v}_{Q}$, and all higher derivatives from differentiation of $x_Q$ \\
We can simply derive the input $f$ from the equation of motion:
\begin{align*}
f = m_Q (\dot{v}_Q + g)
\end{align*}







\end{document}