\documentclass[11pt]{article}
\usepackage[margin=1in]{geometry}

\usepackage{graphicx}
\usepackage{amsfonts}
\usepackage{amssymb}
\usepackage{amsmath}
\usepackage{amsthm}
\usepackage{latexsym}
\usepackage{graphicx}
\usepackage{algorithm} 
\usepackage{algorithmic} 
\usepackage{setspace}
\usepackage{subfig}

\begin{document}

\centering
Trajectory Generation \\
July 31, 2013

\raggedright

\begin{table} [h!]
\footnotesize
\begin{tabular}{ c c }
	$r$ & derivative to minimize in cost function \\
		& implies $r$ initial conditions at each keyframe (position and $r-1$ derivatives, indexed from 0 (constant term) \\
	$n$ & order of desired trajectory, minimum order is $2r-1$ \\
		&  implies $n+1$ coefficients, indexed from 0 (constant term) \\
	$d$ & number of dimensions to optimize, indexed from 1 \\
	 	& for example, $d=3$ when optimizing $[x y z]$ position of a trajectory,  \\
	$m$ & number of pieces in trajectory \\
		& implies $m+1$ keyframes in trajectory, indexed from 1 \\
	$t_{des}$ & vertical vector of desired arrival times at keyframes, indexed from 0 \\
	$pos_{des}$ & matrix of desired positions, each row represents a derivative, each column represents a keyframe \\
		&  Inf represents unconstrained  \\
\end{tabular}
%\caption{Variables used}
\label{tab: vars}
\end{table}




%%%%%%%
\newpage
\small






%%%
% finding minimum order trajectories
\section{Minimum order polynomial trajectories}

Suppose we want to find x(t) that minimizes the functional
\begin{align*}
J &= \int_{t_0}^{t_1} F(x, \dot{x}, ..., x^{(r)}) dt = \int_{t_0}^{t_1} \|  \frac{d^{r} x(t) }{dt} \|^2 dt,
\end{align*}

where $x(t)$ has a polynomial basis:
\begin{align*}
x(t) &= \sum_{i = 0}^{n} c_i t^i
\end{align*}

The functional is minimized by the Lagrangian:
\begin{align*}
\mathcal{L} &= \frac{ \partial F}{\partial x} - \frac{d}{dt} ( \frac{ \partial F}{\partial \dot{x} }  ) + \frac{d^2}{dt^2} ( \frac{ \partial F}{\partial \ddot{x} }  ) - ... + \frac{d^r}{dt^r} ( \frac{ \partial F}{\partial x^{(r)} }  ) \\
&= 0 - 0 + 0 - ... + \frac{d^r}{dt^r} (2 x^{(r)} ) \\
&= 2 x^{(2r)} \\
&= 0 \\
x^{(2r)} &= 0
\end{align*}

To satisfy $x^{(2r)} = 0$, $x(t)$ must have its first $2r-1$ derivatives defined. For a polynomial basis, this implies that $x(t)$ must be at least of order $2r-1$. 

\section{Non-dimensionalization in time}

For a trajectory $X(t)$ from $t_0$ to $t_1$, we can alternatively find the non-dimensionalized $x(\tau)$ from $\tau_0 = 0$ to $\tau_1 = 1$ and scale the result to any $t_0$ and $t_1$ value. For position, this is simply: 
%%
\begin{align*}
\tau &= \frac{t-t_0}{t_1-t_0} \\
X(t) &= x(\tau) = x( \frac{t-t_0}{t_1-t_0}) \\
\frac{d}{dt} X(t) &= \frac{d}{dt} x(\tau) \\
& = \frac{d}{d \tau} \frac{d \tau}{dt} x(\tau) \\
&= \frac{1}{t_1-t_0} \frac{d}{d \tau} x(\tau) \\ 
& ... \\
\frac{d^r}{dt^r} X(t) &= \frac{1}{(t_1-t_0)^r} \frac{d^r}{d \tau^r} x(\tau)
\end{align*}

We can thus solve for each piece of any piece-wise trajectory from $\tau = 0-1$, then scale it for any $t_0$ to $t_1$. Calculating trajectories from $\tau_0 = 0$ to $\tau_1 = 1$ provides more numerical stability. 









%%%
% two keyframes, one dimension
\newpage
\section{Optimization of a trajectory between two keyframes} 

\mbox{} \newline
We seek the desired trajectory $X(t)$. Here, $d = 1$ and $m = 1$. We want to minimize the cost function:

\begin{align*}
J &= \int_{t_0}^{t_1} \|  \frac{d^{r} X(t) }{dt} \|^2 dt  \\
&= X^T Q_{(t0, t1)} X \\
\text{subject to: } A_t X &= b_t
%%
\end{align*}

We can instead look for the non-dimensionalized trajectory $x(\tau) = c_n \tau^n + c_{n-1} \tau^{n-1} + ... c_1 \tau + c_0$, where $\tau = \frac{t-t_0}{t_1-t_0}$. Note that this makes $\tau$ range from $\tau_0 = 0$ to $\tau_1 = 1$. Let $x = [c_n \ \ c_{n-1} \ \ c_{n-2} \ \ ... \ \ c_1 \ \ c_0]^T$. 

\mbox{} \newline
We can write the cost function $J$ in terms of the non-dimsionalized trajectory $x(\tau)$: 
%%
\begin{align*}
J &= \int_{t_0}^{t_1} \|  \frac{d^{r} X(t) }{dt} \|^2 dt  \\
&= \int_{0}^{1} \|  \frac{1}{(t_1-t_0)^r} \frac{d^{r} x(\tau) }{d\tau} \|^2 d \tau  \\
&=  \frac{1}{(t_1-t_0)^{2r}} \int_{0}^{1} \| \frac{d^{r} x(\tau) }{d\tau} \|^2 d \tau \\
&= \frac{1}{(t_1-t_0)^{2r}} x^T Q_{(0, 1)} x \\
&= x^T \left( \frac{1}{(t_1-t_0)^{2r}} Q_{(0, 1)}\right) x
\end{align*}
%%
%Minimizing $\int_{t_0}^{t_1} \|  \frac{d^{r} x(t) }{dt} \|^2 dt$ equates to minimizing $\int_{0}^{1} \| \frac{d^{r} x(\tau) }{d\tau} \|^2 d \tau$, since $\frac{1}{(t_1-t_0)^{2k}}$ is just a constant. 

Thus, we want to minimize the cost function: 
\begin{align*}
J &= x^T \left( \frac{1}{(t_1-t_0)^{2r}} Q_{(0, 1)}\right) x \\
\text{subject to: } A x &= b
%%
\end{align*}

%%%
\mbox{} \newline
\mbox{} \newline
TO FIND Q: \newline
When $x' = [c_0 \ \ c_1 \ \ ... \ \ c_{n-1} \ \ c_n]^T$, we can find $Q'_{(0, 1)}$ with: 
\begin{align}
\label{eqn: Q}  Q'[i, j]_{(t_0, t_1)} &= 
\begin{cases}
    \prod_{k = 0}^{r-1} {(i-k)(j-k)} \frac{ t_1^{i+j-2r+1} - t_0^{i_j-2r+1} }{i+j-2r+1}, & i \ge r \land j \ge r \\
    0, & i < r \lor j < r 
\end{cases}, i = 0...n, j = 0...n
\end{align}

However, our $x = [c_n \ \ c_{n-1} \ \ c_{n-2} \ \ ... \ \ c_1 \ \ c_0]^T$. Reflecting $Q'$ from Eq. \ref{eqn: Q} horizontally and vertically will give us the desired $Q$ for the form of $x$ we desire. The function we want to minimize is then $\left( \frac{1}{(t_1-t_0)^{2r}} Q_{(0, 1)}\right)$. 



%%%
\mbox{} \newline
\mbox{} \newline
TO FIND A: \newline
\begin{align*}
A_t X &= b_t \\
 \begin{bmatrix}
  A(t_0) \\
  A(t_1) \\
 \end{bmatrix} 
 x 
 &= 
\begin{bmatrix}
  X(t_0) \\
  \dot{X} (t_0) \\
  ... \\
  X^{(r-1)}(t_0) \\
    X(t_1) \\
  \dot{X} (t_1) \\
  ... \\
  X^{(r-1)}(t_1) \\
 \end{bmatrix} 
\end{align*}

Note that $Ax$ only contains rows where constraints are specified - omit rows where a condition is unconstrained. Assuming that every condition is constrained, the general form of A is:
\begin{align}
\label{eqn: A} A[i, j] (t) &= 
\begin{cases}
    \prod_{k=0}^{i-1} {(n-k-j)} t^{n-j-i}, & n-j \ge i \\
    0, & n-j < i
\end{cases}, i = 0...(r-1), j = 0...n
\end{align}
%%
where $A[i, j]$ represents the $(n-j)$th coefficient of the $i$th derivative. 

\mbox{} \newline
In the non-dimensionalized case, we have, where $\tau_0 = 0$ and $\tau_1 = 1$:
\begin{align*}
 \begin{bmatrix}
  A(\tau_0) \\
  A(\tau_1) \\
 \end{bmatrix} 
 x
 &= 
\begin{bmatrix}
  X(t_0) \\
  {(t_1-t_0)} \dot{X} (t_0) \\
  ... \\
  {(t_1-t_0)^{r-1}} X^{(r-1)}(t_0) \\
    X(t_1) \\
  {(t_1-t_0)} \dot{X} (t_1) \\
  ... \\
  {(t_1-t_0)^{r-1}} X^{(r-1)}(t_1) \\
 \end{bmatrix}
\end{align*}




%%%
\mbox{} \newline
\mbox{} \newline
TO EVALUATE: \newline

\begin{align*}
X(t) &= 
\begin{cases}
    x(0), & t \le t < t_0 \\
    x(\tau), & t_0 \le t < t_1, \tau = \frac{t-t_0}{t_1-t_0} \\
    x(1), & t \ge t_1 \\
\end{cases}
\end{align*} 

\begin{align*}
X^{(k)}(t) &= 
\begin{cases}
    \frac{1}{(t_1-t_0)^k} x^{(k)}(0), & t < t_0 \\
    \frac{1}{(t_1-t_0)^k} x^{(k)}(\tau), & t_0 \le t < t_1, \tau = \frac{t-t_0}{t_1-t_0} \\
    \frac{1}{(t_1-t_0)^k} x^{(k)}(1), &  t_1 \le t \\
\end{cases}
\end{align*} 











%%%
% m keyframes, one dimension
\newpage
\section{Optimization of a trajectory between $m+1$ keyframes in one dimension} \label{sec: mkeyframes1d}

\mbox{} \newline
We seek the piece-wise trajectory: 
\begin{align*}
X(t) &= 
\begin{cases}
    X_1 (t), & t_0 \le t < t_1 \\
    X_2 (t), & t_1 \le t < t_2 \\
    ... \\
    X_m (t), & t_{m-1} \le t < t_m \\
\end{cases}
\end{align*} 
We continue to minimize the cost function:

\begin{align*}
J &= \int_{t_0}^{t_m} \|  \frac{d^{r} X(t) }{dt} \|^2 dt \\
&= X^T Q_{(t_m, t_0)} X \\
\text{subject to: } A_t X &=b_t
%%
\end{align*}

We again look for the non-dimensionalized trajectory: 
\begin{align*}
x(\tau) &= 
\begin{cases}
    x_1 (\tau) = c_{1, n} \tau^n + c_{1, n-1} \tau^{n-1} + ... c_{1, 1} \tau + c_{1, 0}, & t_0 \le t < t_1, \tau = \frac{t-t_0}{t_1-t_0}  \\
    x_2 (\tau) = c_{2, n} \tau^n + c_{2, n-1} \tau^{n-1} + ... c_{2, 1} \tau + c_{2, 0}, & t_1 \le t < t_2, \tau = \frac{t-t_1}{t_2-t_1}  \\
    ... \\
    x_m (\tau) = c_{m, n} \tau^n + c_{m, n-1} \tau^{n-1} + ... c_{m, 1} \tau + c_{m, 0}, & t_{m-1} \le t < t_m, \tau = \frac{t-t_{m-1}}{t_m-t_{m-1}} \\
\end{cases},  0 \le \tau < 1
\end{align*} 
Let $x_k = [c_{k, n} \ \ c_{k, n-1} \ \ ... \ \ c_{k, 1} \ \ c_{k, 0}]^T$ and $x = [x_1; x_2; ...; x_m] = [c_{1, n} \ \ c_{1, n-1} \ \ c_{1, n-2} \ \ ... \ \ c_{1, 1} \ \ c_{1, 0} \ \ c_{2, n} \ \ c_{2, n-1} \ \ ... \ \ c_{m, 1} \ \ c_{m, 0} ]^T$. Here, $d = 1$. Each piece of the trajectory is individually optimized between $\tau_0 = 0$ and $\tau_1=1$. We evaluate a time $t$ on trajectory $x_k(\tau)$ by finding $k$ such that $t_{k-1} \le t < t_k$ at time $\tau = \frac{t-t_{k-1}}{t_k-t_{k-1}}$.  

\mbox{} \newline
We want to minimize:
\begin{align*}
J &= \int_{t_0}^{t_m} \|  \frac{d^{r} X(t) }{dt} \|^2 dt \\
&= \sum_{k=1}^{m} \int_{t_{k-1}}^{t_k} \|  \frac{d^{r} X_k (t) }{dt} \|^2 dt  \\
&= \sum_{k=1}^{m} \int_{0}^{1} \frac{1}{(t_k-t_{k-1})^{2r}} \|  \frac{d^{r} x_k (\tau) }{d\tau} \|^2 d\tau  \\
&=  \sum_{k=1}^{m} x_k^T \frac{1}{(t_k-t_{k-1})^{2r}} Q_{(0, 1)} x_k \\
&= x^T Q x \\
\text{subject to: } A x &=b
%%
\end{align*}

\mbox{} \newline
Note that alternatively, we could have non-dimensionalized the entire trajectory between 0 and 1 and evaluate a time $t$ on trajectory $x_k(\tau)$ by finding $k$ such $\frac{t_{k-1}-t_0}{t_m-t_0} \le \tau < \frac{t_{k}-t_0}{t_m-t_0}$, where $\tau = \frac{t-t_0}{t_m-t_0}$. 



%%%
\mbox{} \newline
\mbox{} \newline
TO FIND Q: \newline
Recall that for each $x'_k = [c_{k, 0} \ \ c_{k, 1} \ \ ... \ \ c_{k, n-1} \ \ c_{k, n}]^T$, where $k = 1...m$, $Q'_{(0, 1)}$ is given by Eq. \ref{eqn: Q}. Since our $x_k = [c_{k, n} \ \ c_{k, n-1} \ \ ... \ \ c_{k, 1} \ \ c_{k, 0}]^T$, reflecting $Q'$ horizontally and vertically will give us the desired $Q$ for the form of $x_k$. We can then create the block diagonal matrix:
\begin{align}
\label{eqn: Qkeyframes} Q &= 
\begin{bmatrix}
  \frac{1}{(t_1-t_{0})^{2r}} Q_{(0, 1)} & 0 & 0 & ... & 0 \\
  0 & \frac{1}{(t_2-t_{1})^{2r}} Q_{(0, 1)} & 0 & ... & 0 \\
  & & ... & &  \\
  0 & ... & 0 & \frac{1}{(t_{m-1}-t_{m-2})^{2r}} Q_{(0, 1)} & 0 \\
  0 & ... & 0 & 0 & \frac{1}{(t_m-t_{m-1})^{2r}} Q_{(0, 1)} \\ 
 \end{bmatrix}
\end{align}



%%%
\mbox{} \newline
\mbox{} \newline
TO FIND A: \newline

\mbox{} \newline
First, we need to account for endpoint constraints: 
\begin{align*}
A_{endpoint_t} X &= b_{endpoint_t} \\
\begin{bmatrix}
 A(t_0) & 0 & 0 & ... & 0 \\
 A(t_1) & 0 & 0 & ... & 0 \\
 0 & A(t_1) & 0 & ... & 0 \\
 0 & A(t_2) & 0 & ... & 0 \\
 0 & 0 & A(t_2) & ... & 0 \\
 0 & 0 & A(t_3) & ... & 0 \\
 & & ... & & \\
 0 & 0 & ... & 0 & A(t_{m-1}) \\
 0 & 0 & ... & 0 & A(t_m)  \\
 \end{bmatrix}
 X 
 &= 
 \begin{bmatrix}
  X_1 (t_0) \\
  \dot{X}_1 (t_0) \\
  ... \\
  X^{(r-1)}_1 (t_0) \\
    X_1 (t_1) \\
  \dot{X}_1 (t_1) \\
  ... \\
  X^{(r-1)}_1 (t_1) \\
     X_2 (t_1) \\
  \dot{X}_2 (t_1) \\
  ... \\
  X^{(r-1)}_2 (t_1) \\
     X_2 (t_2) \\
  \dot{X}_2 (t_2) \\
  ... \\
  X^{(r-1)}_2 (t_2) \\
  ... \\
     X_m (t_{m-1}) \\
  \dot{X}_m (t_{m-1}) \\
  ... \\
  X^{(r-1)}_m (t_{m-1}) \\
     X_m (t_m) \\
  \dot{X}_m (t_m) \\
  ... \\
  X^{(r-1)}_m (t_m) \\
 \end{bmatrix} 
\end{align*}

In the non-dimensionalized case, we have, $\tau_0=0$, $\tau_1=1$, and:
\begin{align}
\nonumber A_{endpoint} x &= b_{endpoint} \\
\label{eqn: endpoint} \begin{bmatrix}
 A(\tau_0) & 0 & 0 & ... & 0 \\
 A(\tau_1) & 0 & 0 & ... & 0 \\
 0 & A(\tau_0) & 0 & ... & 0 \\
 0 & A(\tau_1) & 0 & ... & 0 \\
 0 & 0 & A(\tau_0) & ... & 0 \\
 0 & 0 & A(\tau_1) & ... & 0 \\
 & & ... & & \\
 0 & 0 & ... & 0 & A(\tau_0) \\
 0 & 0 & ... & 0 & A(\tau_1)  \\
 \end{bmatrix}
 x 
 &= 
 \begin{bmatrix}
  X_1 (t_0) \\
  (t_1-t_0) \dot{X}_1 (t_0) \\
  ... \\
  (t_1-t_0)^{(r-1)}  X^{(r-1)}_1 (t_0) \\
    X_1 (t_1) \\
  (t_1-t_0) \dot{X}_1 (t_1) \\
  ... \\
  (t_1-t_0)^{(r-1)} X^{(r-1)}_1 (t_1) \\
     X_2 (t_1) \\
 (t_2-t_1) \dot{X}_2  (t_1) \\
  ... \\
 (t_2-t_1)^{(r-1)} X^{(r-1)}_2 (t_1) \\
     X_2 (t_2) \\
 (t_2-t_1)  \dot{X}_2 (t_2) \\
  ... \\
 (t_2-t_1)^{(r-1)}  X^{(r-1)}_2 (t_2) \\
  ... \\
     X_m (t_{m-1}) \\
 (t_m-t_{m-1}) \dot{X}_m  (t_{m-1}) \\
  ... \\
(t_m-t_{m-1})^{(r-1)}  X^{(r-1)}_m (t_{m-1}) \\
     X_m (t_m) \\
(t_m-t_{m-1})  \dot{X}_m  (t_m) \\
  ... \\
(t_m-t_{m-1})^{(r-1)}  X^{(r-1)}_m (t_m) \\
 \end{bmatrix} 
\end{align}

Note that again, we omit rows where a condition is unconstrained. Also, except for constraints at $t_0$ and $t_m$, every other constraint must be included twice - a constraint at $t_k$ must be applied as a final condition to $x_{k}(\tau_1)$ and an initial condition $x_{k+1}(\tau_0)$. The equation for $A[i, j] (t)$ is given in Eq. \ref{eqn: A}. 

\mbox{} \newline
We must also account for continuity constraints, which ensure that when the trajectory switches from one piece to another at the keyframes, position and all derivatives lower than $r$ remain continuous, for a smooth path. In other words, we require:
\begin{align*}
A_{cont_t} X &= b_{cont_t} \\
\begin{bmatrix}
  X_1 (t_1) - X_2(t_1) \\
  \dot{X}_1 (t_1) - \dot{X}_2 (t_1) \\
  ... \\
  X^{(r-1)}_1 (t_1) - X^{(r-1)}_2 (t_1) \\
    ... \\
  X_{m-1} (t_{m-1}) - X_{m} (t_{m-1}) \\
  \dot{X}_{m-1} (t_{m-1}) - \dot{X}_{m} (t_{m-1}) \\
  ... \\
  X^{(r-1)}_{m-1} (t_{m-1}) - X^{(r-1)}_m (t_{m-1}) \\
 \end{bmatrix} 
 &=
 0 
 \end{align*}
 
 Translating to the nondimensionalized case, $\tau_0 = 0$, $\tau_1 = 1$, and: 
 \begin{align*}
 A_{cont} x &= b_{cont} \\
 \begin{bmatrix}
  x_1 (\tau_1) - x_2(\tau_0) \\
  \frac{1}{(t_1-t_0)} \dot{x}_1 (\tau_1) - \frac{1}{(t_2-t_1)} \dot{x}_2(\tau_0) \\
  ... \\
  \frac{1}{(t_1-t_0)^{(r-1)}} x^{(r-1)}_1 (\tau_1) - \frac{1}{(t_2-t_1)^{(r-1)}} x^{(r-1)}_2 (\tau_0) \\
    ... \\
  x_{m-1} (\tau_1) - x_{m} (\tau_0) \\
 \frac{1}{(t_{m-2}-t_{m-1})} \dot{x}_{m-1} (\tau_1) -  \frac{1}{(t_m-t_{m-1})}  \dot{x}_{m} (\tau_0) \\
  ... \\
  \frac{1}{(t_{m-2}-t_{m-1})^{(r-1)}} x^{(r-1)}_{m-1} (\tau_1) - \frac{1}{(t_m-t_{m-1})^{(r-1)}} x^{(r-1)}_m (\tau_0) \\
 \end{bmatrix} 
 &=
 0 \\
 \begin{bmatrix}
 A_{cont} (t_1) & 0 & 0 & ... & 0 \\
 0 & A_{cont} (t_2) & 0 & ... & 0 \\
 0 & 0 & A_{cont} (t_3) & ... & 0 \\
 & & ... & & \\
 0 & 0 & ... & 0 & A_{cont} (t_{m-1}) \\
  \end{bmatrix} x 
  &= 
  0
\end{align*} where:
%%%
\begin{align}
\label{eqn: Acont} A_{cont}[i, j] (t_k) &= 
\begin{cases}
   \frac{1}{(t_{k}-t_{k-1})^i} \prod_{k=0}^{i-1} {(n-k-j)} \tau_1^{n-j-i}, & n-j \ge i \land j \le n \\
    0, & n-j < i \land j \le n \\
     - \frac{1}{(t_{k+1}-t_k)^i} \prod_{k=0}^{i-1} {(1-k-j)} \tau_0^{1-j-i}, & 1-j \ge i \land j > n \\   %replace all j with j-n-1 so index of n+1 will translate to j
     0, & 1-j < i \land j > n \\       
\end{cases}, i = 0...(r-1), j = 0...2(n+1)
\end{align}

Our constraints, $Ax = b$, take the form:
\begin{align}
\nonumber Ax &= b \\
\label{eqn: Akeyframes} 
\begin{bmatrix}
A_{endpoint} \\
A_{cont} \\
 \end{bmatrix}
 x 
& = 
 \begin{bmatrix}
b_{endpoint} \\
0 \\
 \end{bmatrix} 
\end{align}



%%%
\mbox{} \newline
\mbox{} \newline
TO EVALUATE: \newline

\begin{align*}
X(t) &= 
\begin{cases}
    x_1(0), & t < t_0 \\
    x_1 (\tau) = c_{1, n} \tau^n + c_{1, n-1} \tau^{n-1} + ... c_{1, 1} \tau + c_{1, 0}, & t_0 \le t < t_1, \tau = \frac{t-t_0}{t_1-t_0}  \\
    x_2 (\tau) = c_{2, n} \tau^n + c_{2, n-1} \tau^{n-1} + ... c_{2, 1} \tau + c_{2, 0}, & t_1 \le t < t_2, \tau = \frac{t-t_1}{t_2-t_1}  \\
    ... \\
    x_m (\tau) = c_{m, n} \tau^n + c_{m, n-1} \tau^{n-1} + ... c_{m, 1} \tau + c_{m, 0}, & t_{m-1} \le t < t_m, \tau = \frac{t-t_{m-1}}{t_m-t_{m-1}} \\
    x_m(1), & t_m \le t \\
\end{cases}
\end{align*} 

\begin{align*}
X^{(k)}(t) &= 
\begin{cases}
    \frac{1}{(t_1-t_0)^k} x^{(k)}_1(0), & t < t_0 \\
    \frac{1}{(t_1-t_0)^k} x^{(k)}_1 (\tau) , & t_0 \le t < t_1, \tau = \frac{t-t_0}{t_1-t_0}  \\
    \frac{1}{(t_2-t_1)^k} x^{(k)}_2 (\tau) , & t_1 \le t < t_2, \tau = \frac{t-t_1}{t_2-t_1}  \\
    ... \\
    \frac{1}{(t_m-t_{m-1})^k} x^{(k)}_m (\tau) , & t_{m-1} \le t < t_m, \tau = \frac{t-t_{m-1}}{t_m-t_{m-1}} \\
    \frac{1}{(t_m-t_{m-1})^k} x^{(k)}_m(1), & t_m \le t \\
\end{cases}
\end{align*} 











%%%
% m keyframes, d dimensions
\newpage
\section{Optimization of a trajectory between $m+1$ keyframes in $d$ dimensions, with corridor constraints} 

\mbox{} \newline
We seek the piecewise multi-dimension trajectory: 
%%
\begin{align*}
\mathbf{X} (t) &= [X_1(t) \ \ X_2(t) \ \ X_3(t) \ \ .... \ \ X_d(t) ], \\
\text{where: } X_p(t) &= 
\begin{cases}
    X_{p, 1} (t), & t_0 \le t < t_1 \\
    X_{p, 2} (t), & t_1 \le t < t_2 \\
    ... \\
    X_{p, m} (t), & t_{m-1} \le t < t_m \\
\end{cases}
\end{align*} 
%%
that minimizes the cost function:

\begin{align*}
J &= \int_{t_0}^{t_m} \|  \frac{d^{r} \mathbf{X} (t) }{dt} \|^2 dt \\
\text{subject to: } A_{eq_t} X &= b_{eq_t} \\
A_{ineq_t} X &\le b_{ineq_t}
%%
\end{align*}



\mbox{} \newline
We can look for the nondimensionalized trajectory: 
%%%
\begin{align*}
\mathbf{X} (\tau) &= [X_1(\tau) \ \ X_2(\tau) \ \ X_3(\tau) \ \ .... \ \ X_d(\tau) ], \\
\text{where: } X_p(\tau) &= 
\begin{cases}
    x_{p, 1} (\tau) = c_{p, 1, n} \tau^n + c_{p, 1, n-1} \tau^{n-1} + ... c_{p, 1, 1} \tau + c_{p, 1, 0}, & t_0 \le t < t_1, \tau = \frac{t-t_0}{t_1-t_0} \\
    x_{p, 2} (\tau) = c_{p, 2, n} \tau^n + c_{p, 2, n-1} \tau^{n-1} + ... c_{p, 2, 1} \tau + c_{p, 2, 0}, & t_1 \le t < t_2, \tau = \frac{t-t_1}{t_2-t_1} \\
    ... \\
    x_{p, m} (\tau) = c_{p, m, n} \tau^n + c_{p, m, n-1} \tau^{n-1} + ... c_{p, m, 1} \tau + c_{p, m, 0}, & t_{m-1} \le t < t_m, \tau = \frac{t-t_{m-1}}{t_m-t_{m-1}} \\
\end{cases}
\end{align*} 
%%%
Let $x_p = [c_{p, 1, n} \ \ ... \ \ c_{p, 1, 0} \ \ c_{p, 2, n} \ \ c_{p, 2, n-1} \ \ ... \ \ c_{p, m, 0}]^T$ and $x = [x_1; x_2^T; ...; x_d^T] = [c_{1, 1, n} \ \ ... \ \ c_{1, 1, 0} \ \ c_{1, 2, n} \ \ c_{1, 2, n-1} \ \ ... \ \ c_{1, m, 0} \ \ c_{2, 1, n} \ \ c_{2, 1, n-1} \ \ ... \ \ c_{2, m, 0} \ \ ... \ \ c_{d, m, 0} ]^T$. We want to minimize the cost function:
%%%
\begin{align*}
J &= \int_{t_0}^{t_m} \|  \frac{d^{r} \mathbf{X} (t) }{dt} \|^2 dt \\
&=  \int_{t_0}^{t_m} \|  [ \frac{d^{r} {X}_1(t) }{dt} \ \  \frac{d^{r} {X}_2(t) }{dt} \ \ ... \ \  \frac{d^{r} {X}_d(t) }{dt}] \|^2 dt \\
&= \int_{t_0}^{t_m}  \left( \sqrt{  \left( \frac{d^{r} {X}_1(t) }{dt} \right)^2 + \left(  \frac{d^{r} {X}_2(t) }{dt} \right)^2 + ... + \left( \frac{d^{r} {X}_d(t) }{dt}\right) ^2 } \right) ^2 dt \\
&= \int_{t_0}^{t_m} \left( \left( \frac{d^{r} {X}_1(t) }{dt} \right)^2 + \left(  \frac{d^{r} {X}_2(t) }{dt} \right)^2 + ... + \left( \frac{d^{r} {X}_d(t) }{dt}\right) ^2 \right) dt \\
&= \sum_{p = 0}^{d} \int_{t_0}^{t_m} \left( \frac{d^{r} {X}_p(t) }{dt}\right) ^2 dt \\
&= \sum_{p = 0}^{d} x_p^T Q_p x_p \\
&= x^T Q x \\
\text{subject to: } A_{eq} x &= b_{eq} \\
A_{ineq} x &\le b_{ineq}
%%
\end{align*}


%%%
\mbox{} \newline
\mbox{} \newline
TO FIND Q: \newline
For each dimension, we define $Q_p$ using Eq. \ref{eqn: Qkeyframes}. We then create the block diagonal matrix:
\begin{align}
\label{eqn: Qdim} Q &= 
\begin{bmatrix}
  Q_1 & 0 & 0 & ... & 0 \\
  0 & Q_2 & 0 & ... & 0 \\
  & & ... & &  \\
  0 & ... & 0 & Q_{p-1} & 0 \\
  0 & ... & 0 & 0 & Q_p \\ 
 \end{bmatrix}
\end{align}




%%%
\mbox{} \newline
\mbox{} \newline
TO FIND $A_{eq}$: \newline

We simply use Eq. \ref{eqn: Akeyframes} to find $A_{eq_p} x = b_{eq_p} $ for each dimension and create the block diagonal matrix:
\begin{align}
\label{eqn: Aeqdim} A_{eq} &= 
\begin{bmatrix}
  A_{eq_1} & 0 & 0 & ... & 0 \\
  0 & A_{eq_2} & 0 & ... & 0 \\
  & & ... & &  \\
  0 & ... & 0 & A_{eq_{p-1}} & 0 \\
  0 & ... & 0 & 0 & A_{eq_p} \\ 
 \end{bmatrix} \\
 \nonumber b_{eq} &= 
 \begin{bmatrix}
  b_{eq_1}  \\
  b_{eq_2}  \\
  ... \\
    b_{eq_{p-1}}  \\
      b_{eq_p}  \\
 \end{bmatrix} 
\end{align}





%%%
\mbox{} \newline
\mbox{} \newline
TO FIND $A_{ineq}$: \newline

We can add corridor, or inequality constraints, to the paths between keyframes as well. Let the constraint $i$ be between keyframe $i$ and $i+1$ and applied to dimensions $a$, $b$, and $c$ (for example, if the trajectory dimensions were $[\psi \ \ x \ \ y \ \ \phi \ \ z]^T$, the dimensions $x$, $y$, and $z$ position would be $a = 2$, $b=3$, $c=4$). Let $\mathbf{r}_i = [X_{a, i} \ \ X_{b, i} \ \ X_{c, i}]^T$, or the position vector of keyframe $i$ and $\mathbf{X} (t) = [X_a (t) \ \ X_b (t) \ \ X_{c}(t)]^T$. We want position to stay within a corridor of width $\delta_i$ and imposed using $n_c$ intermediate points. 
%%%
\begin{align*}
\mathbf{t}_i &= \frac{ \mathbf{r}_{i+1} - \mathbf{r}_i } { \| \mathbf{r}_{i+1} - \mathbf{r}_i \| } \\
\mathbf{d}_i (t) &= (\mathbf{X}(t) - \mathbf{r}_i) - ((\mathbf{X}(t) - \mathbf{r}_i) \cdot \mathbf{t}_i) \mathbf{t}_i
\end{align*}

We want to satisfy the constraint:
\begin{align*}
\| \mathbf{d}_i \|_{\infty} & \le \delta_i \text{ for } t_i \le t \le t_{i+1} \\
| \mathbf{e}_p \cdot \mathbf{d}_i (t_i + \frac{j}{1+n_c} (t_{i+1} - t_i) | & \le \delta_i ,  p = a, b, c, j = 1 ... n_c 
\end{align*}

The inequality breaks down into: 
\begin{align*}
(\mathbf{e}_p \cdot \mathbf{d}_i (t_i + \frac{j}{1+n_c} (t_{i+1} - t_i) ) & \le \delta_i \\
- (\mathbf{e}_p \cdot \mathbf{d}_i (t_i + \frac{j}{1+n_c} (t_{i+1} - t_i) ) & \le \delta_i 
\end{align*}

This results in a total of $2(p)(n_c)$ constraints for each corridor constraint. 

\begin{align*}
\mathbf{d}_i (t) &= (\mathbf{X}(t) - \mathbf{r}_i) - ((\mathbf{X}(t) - \mathbf{r}_i) \cdot \mathbf{t}_i) \mathbf{t}_i \\
&= (\mathbf{X}(t) - \mathbf{r}_i) - \left( \sum_{j = 1}^{p} {((\mathbf{X}(t) - \mathbf{r}_i) \cdot \mathbf{e}_j)(\mathbf{t}_i \cdot \mathbf{e}_j)} \right) \mathbf{t}_i \\ 
&= (\mathbf{X}(t) - \mathbf{r}_i) - \left( \sum_{j = 1}^{p}  {((\mathbf{X}(t) - \mathbf{r}_i) \cdot \mathbf{e}_j)((\frac{ \mathbf{r}_{i+1} - \mathbf{r}_i } { \| \mathbf{r}_{i+1} - \mathbf{r}_i \| }) \cdot \mathbf{e}_j)} \right) (\frac{ \mathbf{r}_{i+1} - \mathbf{r}_i } { \| \mathbf{r}_{i+1} - \mathbf{r}_i \| }) \\ 
&= (\mathbf{X}(t) - \mathbf{r}_i) - \left( \sum_{j = 1}^{p}  {((\mathbf{X}(t) - \mathbf{r}_i) \cdot \mathbf{e}_j)(( \mathbf{r}_{i+1} - \mathbf{r}_i ) \cdot \mathbf{e}_j)} \right) (\frac{ \mathbf{r}_{i+1} - \mathbf{r}_i } { \| \mathbf{r}_{i+1} - \mathbf{r}_i \|^2 }) \\ 
\end{align*}

For dimension $\mathbf{e}_a$: 
\begin{align*}
(\mathbf{X}(t) - \mathbf{r}_i) \cdot \mathbf{e}_a - \left( \sum_{j = 1}^{p}  {((\mathbf{X}(t) - \mathbf{r}_i) \cdot \mathbf{e}_j)(( \mathbf{r}_{i+1} - \mathbf{r}_i ) \cdot \mathbf{e}_j)} \right) (\frac{ \mathbf{r}_{i+1} - \mathbf{r}_i } { \| \mathbf{r}_{i+1} - \mathbf{r}_i \|^2 }) \cdot \mathbf{e}_a & \le \delta_i \\
%%%
\mathbf{X}(t) \cdot \mathbf{e}_a - \mathbf{r}_i \cdot \mathbf{e}_a - \left( \sum_{j = 1}^{p}  {(\mathbf{X}(t) \cdot \mathbf{e}_j)(( \mathbf{r}_{i+1} - \mathbf{r}_i ) \cdot \mathbf{e}_j)} \right) (\frac{ \mathbf{r}_{i+1} - \mathbf{r}_i } { \| \mathbf{r}_{i+1} - \mathbf{r}_i \|^2 }) \cdot \mathbf{e}_a &\\
- \left( \sum_{j = 1}^{p}  {(\mathbf{r}_i \cdot \mathbf{e}_j)(( \mathbf{r}_{i+1} - \mathbf{r}_i ) \cdot \mathbf{e}_j)} \right) (\frac{ \mathbf{r}_{i+1} - \mathbf{r}_i } { \| \mathbf{r}_{i+1} - \mathbf{r}_i \|^2 }) \cdot \mathbf{e}_a & \le \delta_i \\
 %%%
 \mathbf{X}(t) \cdot \mathbf{e}_a - \mathbf{r}_i \cdot \mathbf{e}_a - (\mathbf{X}(t) \cdot \mathbf{e}_a) (\frac{ (\mathbf{r}_{i+1} - \mathbf{r}_i)^2 } { \| \mathbf{r}_{i+1} - \mathbf{r}_i \|^2 }) \cdot \mathbf{e}_a  &\\
 - \left( \sum_{j = 1, j \ne a}^{p}  {(\mathbf{X}(t) \cdot \mathbf{e}_j)(( \mathbf{r}_{i+1} - \mathbf{r}_i ) \cdot \mathbf{e}_j)} \right) (\frac{ \mathbf{r}_{i+1} - \mathbf{r}_i } { \| \mathbf{r}_{i+1} - \mathbf{r}_i \|^2 }) \cdot \mathbf{e}_a &\\
- \left( \sum_{j = 1}^{p}  {(\mathbf{r}_i \cdot \mathbf{e}_j)(( \mathbf{r}_{i+1} - \mathbf{r}_i ) \cdot \mathbf{e}_j)} \right) (\frac{ \mathbf{r}_{i+1} - \mathbf{r}_i } { \| \mathbf{r}_{i+1} - \mathbf{r}_i \|^2 }) \cdot \mathbf{e}_a & \le \delta_i \\
 %%%
\mathbf{X}(t) \cdot \mathbf{e}_a \left( 1 - (\frac{ (\mathbf{r}_{i+1} - \mathbf{r}_i)^2 } { \| \mathbf{r}_{i+1} - \mathbf{r}_i \|^2 }) \cdot \mathbf{e}_a \right) - \left( \sum_{j = 1, j \ne a}^{p}  {(\mathbf{X}(t) \cdot \mathbf{e}_j)(( \mathbf{r}_{i+1} - \mathbf{r}_i ) \cdot \mathbf{e}_j)} \right) (\frac{ \mathbf{r}_{i+1} - \mathbf{r}_i } { \| \mathbf{r}_{i+1} - \mathbf{r}_i \|^2 }) \cdot \mathbf{e}_a &\\
  \le \delta_i + \mathbf{r}_i \cdot \mathbf{e}_a +  \left( \sum_{j = 1}^{p}  {(\mathbf{r}_i \cdot \mathbf{e}_j)(( \mathbf{r}_{i+1} - \mathbf{r}_i ) \cdot \mathbf{e}_j)} \right) (\frac{ \mathbf{r}_{i+1} - \mathbf{r}_i } { \| \mathbf{r}_{i+1} - \mathbf{r}_i \|^2 }) \cdot \mathbf{e}_a & 
\end{align*}

\begin{align*}
- \left( (\mathbf{X}(t) - \mathbf{r}_i) \cdot \mathbf{e}_a - \left( \sum_{j = 1}^{p}  {((\mathbf{X}(t) - \mathbf{r}_i) \cdot \mathbf{e}_j)(( \mathbf{r}_{i+1} - \mathbf{r}_i ) \cdot \mathbf{e}_j)} \right) (\frac{ \mathbf{r}_{i+1} - \mathbf{r}_i } { \| \mathbf{r}_{i+1} - \mathbf{r}_i \|^2 }) \cdot \mathbf{e}_a \right) & \le \delta_i \\
%%%
- \mathbf{X}(t) \cdot \mathbf{e}_a \left( 1 - (\frac{ (\mathbf{r}_{i+1} - \mathbf{r}_i)^2 } { \| \mathbf{r}_{i+1} - \mathbf{r}_i \|^2 }) \cdot \mathbf{e}_a \right) + \left( \sum_{j = 1, j \ne a}^{p}  {(\mathbf{X}(t) \cdot \mathbf{e}_j)(( \mathbf{r}_{i+1} - \mathbf{r}_i ) \cdot \mathbf{e}_j)} \right) (\frac{ \mathbf{r}_{i+1} - \mathbf{r}_i } { \| \mathbf{r}_{i+1} - \mathbf{r}_i \|^2 }) \cdot \mathbf{e}_a &\\
 \le \delta_i - \mathbf{r}_i \cdot \mathbf{e}_a -  \left( \sum_{j = 1}^{p}  {(\mathbf{r}_i \cdot \mathbf{e}_j)(( \mathbf{r}_{i+1} - \mathbf{r}_i ) \cdot \mathbf{e}_j)} \right) (\frac{ \mathbf{r}_{i+1} - \mathbf{r}_i } { \| \mathbf{r}_{i+1} - \mathbf{r}_i \|^2 }) \cdot \mathbf{e}_a & 
\end{align*}



We can then construct the $A_{ineq_t}$ and $b_{ineq_t}$ matrices: 
\begin{align*}
A_{ineq_t} x \le b_{ineq_t} \\
\begin{bmatrix}
(\mathbf{e}_a \cdot \mathbf{d}_i (t_i + \frac{1}{1+n_c} (t_{i+1} - t_i) )) \\
-(\mathbf{e}_a \cdot \mathbf{d}_i (t_i + \frac{1}{1+n_c} (t_{i+1} - t_i) )) \\
(\mathbf{e}_a \cdot \mathbf{d}_i (t_i + \frac{2}{1+n_c} (t_{i+1} - t_i) )) \\
-(\mathbf{e}_a \cdot \mathbf{d}_i (t_i + \frac{2}{1+n_c} (t_{i+1} - t_i) )) \\
... \\
(\mathbf{e}_a \cdot \mathbf{d}_i (t_i + \frac{n_c}{1+n_c} (t_{i+1} - t_i) )) \\
-(\mathbf{e}_a \cdot \mathbf{d}_i (t_i + \frac{n_c}{1+n_c} (t_{i+1} - t_i) )) \\
(\mathbf{e}_b \cdot \mathbf{d}_i (t_i + \frac{1}{1+n_c} (t_{i+1} - t_i) )) \\
-(\mathbf{e}_b \cdot \mathbf{d}_i (t_i + \frac{1}{1+n_c} (t_{i+1} - t_i) )) \\
... \\
(\mathbf{e}_c \cdot \mathbf{d}_i (t_i + \frac{n_c}{1+n_c} (t_{i+1} - t_i) )) \\
-(\mathbf{e}_c \cdot \mathbf{d}_i (t_i + \frac{n_c}{1+n_c} (t_{i+1} - t_i) )) \\
\end{bmatrix} 
& \le 
\begin{bmatrix}
\delta_i \\
\delta_i \\
\delta_i \\
\delta_i \\ 
... \\
\delta_i \\
\delta_i \\
\delta_i \\
\delta_i \\ 
... \\ 
\delta_i \\
\delta_i \\
\end{bmatrix} \\
%%%
 \begin{bmatrix}
 \mathbf{X}(t_1) \cdot \mathbf{e}_a \left( 1 - (\frac{ (\mathbf{r}_{i+1} - \mathbf{r}_i)^2 } { \| \mathbf{r}_{i+1} - \mathbf{r}_i \|^2 }) \cdot \mathbf{e}_a \right) - \left( \sum_{j = 1, j \ne a}^{p}  {(\mathbf{X}(t_1) \cdot \mathbf{e}_j)(( \mathbf{r}_{i+1} - \mathbf{r}_i ) \cdot \mathbf{e}_j)} \right) (\frac{ \mathbf{r}_{i+1} - \mathbf{r}_i } { \| \mathbf{r}_{i+1} - \mathbf{r}_i \|^2 }) \cdot \mathbf{e}_a \\
%%%
- \mathbf{X}(t_1) \cdot \mathbf{e}_a \left( 1 - (\frac{ (\mathbf{r}_{i+1} - \mathbf{r}_i)^2 } { \| \mathbf{r}_{i+1} - \mathbf{r}_i \|^2 }) \cdot \mathbf{e}_a \right) + \left( \sum_{j = 1, j \ne a}^{p}  {(\mathbf{X}(t_1) \cdot \mathbf{e}_j)(( \mathbf{r}_{i+1} - \mathbf{r}_i ) \cdot \mathbf{e}_j)} \right) (\frac{ \mathbf{r}_{i+1} - \mathbf{r}_i } { \| \mathbf{r}_{i+1} - \mathbf{r}_i \|^2 }) \cdot \mathbf{e}_a \\
%%%
... \\
%%%
 \mathbf{X}(t_{n_c}) \cdot \mathbf{e}_a \left( 1 - (\frac{ (\mathbf{r}_{i+1} - \mathbf{r}_i)^2 } { \| \mathbf{r}_{i+1} - \mathbf{r}_i \|^2 }) \cdot \mathbf{e}_a \right) - \left( \sum_{j = 1, j \ne a}^{p}  {(\mathbf{X}(t_{n_c}) \cdot \mathbf{e}_j)(( \mathbf{r}_{i+1} - \mathbf{r}_i ) \cdot \mathbf{e}_j)} \right) (\frac{ \mathbf{r}_{i+1} - \mathbf{r}_i } { \| \mathbf{r}_{i+1} - \mathbf{r}_i \|^2 }) \cdot \mathbf{e}_a \\
%%%
- \mathbf{X}(t_{n_c}) \cdot \mathbf{e}_a \left( 1 - (\frac{ (\mathbf{r}_{i+1} - \mathbf{r}_i)^2 } { \| \mathbf{r}_{i+1} - \mathbf{r}_i \|^2 }) \cdot \mathbf{e}_a \right) + \left( \sum_{j = 1, j \ne a}^{p}  {(\mathbf{X}(t_{n_c}) \cdot \mathbf{e}_j)(( \mathbf{r}_{i+1} - \mathbf{r}_i ) \cdot \mathbf{e}_j)} \right) (\frac{ \mathbf{r}_{i+1} - \mathbf{r}_i } { \| \mathbf{r}_{i+1} - \mathbf{r}_i \|^2 }) \cdot \mathbf{e}_a \\
%%%
 \mathbf{X}(t_1) \cdot \mathbf{e}_b \left( 1 - (\frac{ (\mathbf{r}_{i+1} - \mathbf{r}_i)^2 } { \| \mathbf{r}_{i+1} - \mathbf{r}_i \|^2 }) \cdot \mathbf{e}_b \right) - \left( \sum_{j = 1, j \ne a}^{p}  {(\mathbf{X}(t_1) \cdot \mathbf{e}_j)(( \mathbf{r}_{i+1} - \mathbf{r}_i ) \cdot \mathbf{e}_j)} \right) (\frac{ \mathbf{r}_{i+1} - \mathbf{r}_i } { \| \mathbf{r}_{i+1} - \mathbf{r}_i \|^2 }) \cdot \mathbf{e}_b \\
%%%
- \mathbf{X}(t_1) \cdot \mathbf{e}_b \left( 1 - (\frac{ (\mathbf{r}_{i+1} - \mathbf{r}_i)^2 } { \| \mathbf{r}_{i+1} - \mathbf{r}_i \|^2 }) \cdot \mathbf{e}_b \right) + \left( \sum_{j = 1, j \ne a}^{p}  {(\mathbf{X}(t_1) \cdot \mathbf{e}_j)(( \mathbf{r}_{i+1} - \mathbf{r}_i ) \cdot \mathbf{e}_j)} \right) (\frac{ \mathbf{r}_{i+1} - \mathbf{r}_i } { \| \mathbf{r}_{i+1} - \mathbf{r}_i \|^2 }) \cdot \mathbf{e}_b \\
%%%
... \\
%%%
 \mathbf{X}(t_{n_c}) \cdot \mathbf{e}_c \left( 1 - (\frac{ (\mathbf{r}_{i+1} - \mathbf{r}_i)^2 } { \| \mathbf{r}_{i+1} - \mathbf{r}_i \|^2 }) \cdot \mathbf{e}_c \right) - \left( \sum_{j = 1, j \ne a}^{p}  {(\mathbf{X}(t_{n_c}) \cdot \mathbf{e}_j)(( \mathbf{r}_{i+1} - \mathbf{r}_i ) \cdot \mathbf{e}_j)} \right) (\frac{ \mathbf{r}_{i+1} - \mathbf{r}_i } { \| \mathbf{r}_{i+1} - \mathbf{r}_i \|^2 }) \cdot \mathbf{e}_c \\
%%%
- \mathbf{X}(t_{n_c}) \cdot \mathbf{e}_c \left( 1 - (\frac{ (\mathbf{r}_{i+1} - \mathbf{r}_i)^2 } { \| \mathbf{r}_{i+1} - \mathbf{r}_i \|^2 }) \cdot \mathbf{e}_c \right) + \left( \sum_{j = 1, j \ne a}^{p}  {(\mathbf{X}(t_{n_c}) \cdot \mathbf{e}_j)(( \mathbf{r}_{i+1} - \mathbf{r}_i ) \cdot \mathbf{e}_j)} \right) (\frac{ \mathbf{r}_{i+1} - \mathbf{r}_i } { \| \mathbf{r}_{i+1} - \mathbf{r}_i \|^2 }) \cdot \mathbf{e}_c \\
\end{bmatrix} & \\
\le
\begin{bmatrix}
\delta_i + \mathbf{r}_i \cdot \mathbf{e}_a +  \left( \sum_{j = 1}^{p}  {(\mathbf{r}_i \cdot \mathbf{e}_j)(( \mathbf{r}_{i+1} - \mathbf{r}_i ) \cdot \mathbf{e}_j)} \right) (\frac{ \mathbf{r}_{i+1} - \mathbf{r}_i } { \| \mathbf{r}_{i+1} - \mathbf{r}_i \|^2 }) \cdot \mathbf{e}_a \\
%%%
\delta_i - \mathbf{r}_i \cdot \mathbf{e}_a -  \left( \sum_{j = 1}^{p}  {(\mathbf{r}_i \cdot \mathbf{e}_j)(( \mathbf{r}_{i+1} - \mathbf{r}_i ) \cdot \mathbf{e}_j)} \right) (\frac{ \mathbf{r}_{i+1} - \mathbf{r}_i } { \| \mathbf{r}_{i+1} - \mathbf{r}_i \|^2 }) \cdot \mathbf{e}_a  \\
%%%
... \\
\delta_i + \mathbf{r}_i \cdot \mathbf{e}_a +  \left( \sum_{j = 1}^{p}  {(\mathbf{r}_i \cdot \mathbf{e}_j)(( \mathbf{r}_{i+1} - \mathbf{r}_i ) \cdot \mathbf{e}_j)} \right) (\frac{ \mathbf{r}_{i+1} - \mathbf{r}_i } { \| \mathbf{r}_{i+1} - \mathbf{r}_i \|^2 }) \cdot \mathbf{e}_a \\
%%%
\delta_i - \mathbf{r}_i \cdot \mathbf{e}_a -  \left( \sum_{j = 1}^{p}  {(\mathbf{r}_i \cdot \mathbf{e}_j)(( \mathbf{r}_{i+1} - \mathbf{r}_i ) \cdot \mathbf{e}_j)} \right) (\frac{ \mathbf{r}_{i+1} - \mathbf{r}_i } { \| \mathbf{r}_{i+1} - \mathbf{r}_i \|^2 }) \cdot \mathbf{e}_a  \\
%%%
\delta_i + \mathbf{r}_i \cdot \mathbf{e}_b +  \left( \sum_{j = 1}^{p}  {(\mathbf{r}_i \cdot \mathbf{e}_j)(( \mathbf{r}_{i+1} - \mathbf{r}_i ) \cdot \mathbf{e}_j)} \right) (\frac{ \mathbf{r}_{i+1} - \mathbf{r}_i } { \| \mathbf{r}_{i+1} - \mathbf{r}_i \|^2 }) \cdot \mathbf{e}_b \\
%%%
\delta_i - \mathbf{r}_i \cdot \mathbf{e}_b -  \left( \sum_{j = 1}^{p}  {(\mathbf{r}_i \cdot \mathbf{e}_j)(( \mathbf{r}_{i+1} - \mathbf{r}_i ) \cdot \mathbf{e}_j)} \right) (\frac{ \mathbf{r}_{i+1} - \mathbf{r}_i } { \| \mathbf{r}_{i+1} - \mathbf{r}_i \|^2 }) \cdot \mathbf{e}_b  \\
%%%
... \\
%%%
\delta_i + \mathbf{r}_i \cdot \mathbf{e}_c +  \left( \sum_{j = 1}^{p}  {(\mathbf{r}_i \cdot \mathbf{e}_j)(( \mathbf{r}_{i+1} - \mathbf{r}_i ) \cdot \mathbf{e}_j)} \right) (\frac{ \mathbf{r}_{i+1} - \mathbf{r}_i } { \| \mathbf{r}_{i+1} - \mathbf{r}_i \|^2 }) \cdot \mathbf{e}_c \\
%%%
\delta_i - \mathbf{r}_i \cdot \mathbf{e}_c -  \left( \sum_{j = 1}^{p}  {(\mathbf{r}_i \cdot \mathbf{e}_j)(( \mathbf{r}_{i+1} - \mathbf{r}_i ) \cdot \mathbf{e}_j)} \right) (\frac{ \mathbf{r}_{i+1} - \mathbf{r}_i } { \| \mathbf{r}_{i+1} - \mathbf{r}_i \|^2 }) \cdot \mathbf{e}_c  \\
\end{bmatrix} &
\end{align*}

Since these are all position constraints, we can simply use $A_{ineq} = A_{ineq_t}$ where $\mathbf{X}(t_i) = \mathbf{X}(\tau_0 + \frac{i}{1+n_c} (\tau_1-\tau_0))$ and $b_{ineq} = b_{ineq_t}$. 



%%%
\mbox{} \newline
\mbox{} \newline
TO EVALUATE: \newline

\begin{align*}
\mathbf{X}(t) = [X_1(t) \ \ X_2(t) \ \ ... \ \ X_d(t)]^T
\end{align*}

\begin{align*}
X_p(t) &= 
\begin{cases}
    x_{p, 1} (0), & t < t_0 \\
    x_{p, 1} (\tau) = c_{p, 1, n} \tau^n + c_{p, 1, n-1} \tau^{n-1} + ... c_{p, 1, 1} \tau + c_{p, 1, 0}, & t_0 \le t < t_1, \tau = \frac{t-t_0}{t_1-t_0}  \\
    x_{p, 2} (\tau) = c_{p, 2, n} \tau^n + c_{p, 2, n-1} \tau^{n-1} + ... c_{p, 2, 1} \tau + c_{p, 2, 0}, & t_1 \le t < t_2, \tau = \frac{t-t_1}{t_2-t_1}  \\
    ... \\
    x_{p, m} (\tau) = c_{p, m, n} \tau^n + c_{p, m, n-1} \tau^{n-1} + ... c_{p, m, 1} \tau + c_{p, m, 0}, & t_{m-1} \le t < t_m, \tau = \frac{t-t_{m-1}}{t_m-t_{m-1}} \\
    x_{p, m} (1), & t_m \le t \\
\end{cases}
\end{align*} 

\begin{align*}
X_p^{(k)}(t) &= 
\begin{cases}
    \frac{1}{(t_1-t_0)^k} x^{(k)}_{p,1}(0), & t < t_0 \\
    \frac{1}{(t_1-t_0)^k} x^{(k)}_{p,1} (\tau) , & t_0 \le t < t_1, \tau = \frac{t-t_0}{t_1-t_0}  \\
    \frac{1}{(t_2-t_1)^k} x^{(k)}_{p,2} (\tau) , & t_1 \le t < t_2, \tau = \frac{t-t_1}{t_2-t_1}  \\
    ... \\
    \frac{1}{(t_m-t_{m-1})^k} x^{(k)}_{p,m} (\tau) , & t_{m-1} \le t < t_m, \tau = \frac{t-t_{m-1}}{t_m-t_{m-1}} \\
    \frac{1}{(t_m-t_{m-1})^k} x^{(k)}_{p,m} (1), & t_m \le t \\
\end{cases}
\end{align*} 








%%%
% alternate formulation
\newpage
\section{Alternate formation of optimization of $m+1$ keyframes in one dimensions as a joint optimization problem}

\mbox{} \newline
We seek the piece-wise trajectory: 
\begin{align*}
X(t) &= 
\begin{cases}
    X_1 (t), & t_0 \le t < t_1 \\
    X_2 (t), & t_1 \le t < t_2 \\
    ... \\
    X_m (t), & t_{m-1} \le t < t_m \\
\end{cases}
\end{align*} 
We continue to minimize the cost function:

\begin{align*}
J &= \int_{t_0}^{t_m} \|  \frac{d^{r} X(t) }{dt} \|^2 dt \\
&= X^T Q_{(t_m, t_0)} X \\
\text{subject to: } A_t X &=b_t
%%
\end{align*}

We again look for the non-dimensionalized trajectory: 
\begin{align*}
x(\tau) &= 
\begin{cases}
    x_1 (\tau) = c_{1, n} \tau^n + c_{1, n-1} \tau^{n-1} + ... c_{1, 1} \tau + c_{1, 0}, & t_0 \le t < t_1, \tau = \frac{t-t_0}{t_1-t_0}  \\
    x_2 (\tau) = c_{2, n} \tau^n + c_{2, n-1} \tau^{n-1} + ... c_{2, 1} \tau + c_{2, 0}, & t_1 \le t < t_2, \tau = \frac{t-t_1}{t_2-t_1}  \\
    ... \\
    x_m (\tau) = c_{m, n} \tau^n + c_{m, n-1} \tau^{n-1} + ... c_{m, 1} \tau + c_{m, 0}, & t_{m-1} \le t < t_m, \tau = \frac{t-t_{m-1}}{t_m-t_{m-1}} \\
\end{cases},  0 \le \tau < 1
\end{align*} 
Let $x_k = [c_{k, n} \ \ c_{k, n-1} \ \ ... \ \ c_{k, 1} \ \ c_{k, 0}]^T$ and $x = [x_1; x_2; ...; x_m] = [c_{1, n} \ \ c_{1, n-1} \ \ c_{1, n-2} \ \ ... \ \ c_{1, 1} \ \ c_{1, 0} \ \ c_{2, n} \ \ c_{2, n-1} \ \ ... \ \ c_{m, 1} \ \ c_{m, 0} ]^T$. Here, $d = 1$. Each piece of the trajectory is individually optimized between $\tau_0 = 0$ and $\tau_1=1$. We evaluate a time $t$ on trajectory $x_k(\tau)$ by finding $k$ such that $t_{k-1} \le t < t_k$ at time $\tau = \frac{t-t_{k-1}}{t_k-t_{k-1}}$.  

\mbox{} \newline
We want to minimize:
\begin{align*}
J &=  \sum_{k=1}^{m} x_k^T \frac{1}{(t_k-t_{k-1})^{2r}} Q_{(0, 1)} x_k \\
&= x^T Q x \\
\text{subject to: } A x &=b
%%
\end{align*}




%\mbox{} \newline
%We seek the piece-wise trajectory: 
%\begin{align*}
%x(t) &= 
%\begin{cases}
%    x_1 (t) = c_{1, n} t^n + c_{1, n-1} t^{n-1} + ... c_{1, 1} t + c_{1, 0}, & t_0 \le t < t_1 \\
%    x_2 (t) = c_{2, n} t^n + c_{2, n-1} t^{n-1} + ... c_{2, 1} t + c_{2, 0}, & t_1 \le t < t_2 \\
%    ... \\
%    x_m (t) = c_{m, n} t^n + c_{m, n-1} t^{n-1} + ... c_{m, 1} t + c_{m, 0}, & t_{m-1} \le t < t_m \\
%\end{cases}
%\end{align*} 
%Let $x = [c_{1, n} \ \ c_{1, n-1} \ \ c_{1, n-2} \ \ ... \ \ c_{1, 1} \ \ c_{1, 0} \ \ c_{2, n} \ \ c_{2, n-1} \ \ ... \ \ c_{m, 1} \ \ c_{m, 0} ]^T$. %Here, $d = 1$. We continue to minimize the cost function:
%
%\begin{align*}
%J &= \int_{t_0}^{t_m} \|  \frac{d^{r} x(t) }{dt} \|^2 dt = x^T Q x
%%%
%\end{align*}



%%%
\mbox{} \newline
\mbox{} \newline
TO FIND Q: \newline
We define $Q$ using the block diagonal matrix found in Eq. \ref{eqn: Qkeyframes}. 



%%%
\mbox{} \newline
\mbox{} \newline
TO FIND A: \newline
We create the alternate state 
\begin{align*}
d &= [X_1(t_0) \ \ \dot{X}_1(t_0) \ \ ... \ \ X^{(r-1)}_1(t_0) \\
& \ \  X_1(t_1) \ \ \dot{X}_1(t_1) \ \ ... \ \ X^{(r-1)}_1(t_1) \\
& \ \ X_2(t_1) \ \ \dot{X}_2(t_1) \ \ ... \ \ X^{(r-1)}_2(t_1) \\  
& \ \ X_2(t_2) \ \ \dot{X}_2(t_2) \ \ ... \ \ X^{(r-1)}_2(t_2) \ \ ...  \\
& \ \ X_m (t_m) \ \ \dot{X}_m(t_m) \ \ ... \ \ X^{(r-1)}_m(t_m)]^T \\
&= [x_1(\tau_0) \ \ \frac{1}{t_1-t_0} \dot{x}_1(\tau_0) \ \ ... \ \ \frac{1}{(t_1-t_0)^{(r-1)}} x^{(r-1)}_1(\tau_0) \\
& \ \  x_1(\tau_1) \ \ \frac{1}{t_1-t_0} \dot{x}_1 (\tau_1) \ \ ... \ \ \frac{1}{(t_1-t_0)^{(r-1)}} x^{(r-1)}_1(\tau_1) \\
& \ \ x_2(\tau_0) \ \ \frac{1}{t_2-t_1} \dot{x}_2 (\tau_0) \ \ ... \ \ \frac{1}{(t_2-t_1)^{(r-1)}} x^{(r-1)}_2(\tau_0) \\
& \ \ x_2(\tau_1) \ \ \frac{1}{t_2-t_1} \dot{x}_2 (\tau_1) \ \ ... \ \ \frac{1}{(t_2-t_1)^{(r-1)}} x^{(r-1)}_2(\tau_1) \ \ ...  \\
& \ \ x_m (\tau_1) \ \ \frac{1}{t_m-t_{m-1}} \dot{x}_m (\tau_1) \ \ ... \ \ \frac{1}{(t_m-t_{m-1})^{(r-1)}} x^{(r-1)}_m(\tau_1)]^T \\
\end{align*}

Here, we define matrix $A_k$ such that:
\begin{align*}
d_k &= [X_k(t_{k-1}) \ \ \dot{X}_k(t_{k-1}) \ \ ... \ \ X^{(r-1)}_k(t_{k-1}) \ \  X_k(t_k) \ \ \dot{X}_k(t_k) \ \ ... \ \ X^{(r-1)}_k(t_k)]^T  \\
& = [x_k(\tau_0) \ \ \frac{1}{t_k-t_{k-1}} \dot{x}_k(\tau_0) \ \ ... \ \ \frac{1}{(t_k-t_{k-1})^{(r-1)}} x^{(r-1)}_k(\tau_0) \ \  x_k(\tau_1) \ \ \frac{1}{t_k-t_{k-1}} \dot{x}_k(\tau_1) \ \ ... \ \ \frac{1}{(t_k-t_{k-1})^{(r-1)}} x^{(r-1)}_k(\tau_1)]^T \\
&= A_k x_k \\
&= 
\begin{bmatrix}
A_k(\tau_0) \\
A_k(\tau_1)
\end{bmatrix}
 [c_{k, n} \ \ c_{k, n-1} \ \ c_{k, n-2} \ \ ... \ \ c_{k, 1} \ \ c_{k, 0}]^T 
\end{align*}

where:

\begin{align*}
A_k[i, j] (t) &= 
\begin{cases}
    \frac{1}{(t_k-t_{k-1})^i} \prod_{k=0}^{i-1} {(n-k-j)} t^{n-j-i}, & n-j \ge i \\
    0, & n-j < i
\end{cases}, i = 0...(r-1), j = 0...n
\end{align*}

We can then construct matrix $A$ where $d = Ax$:
\begin{align}
\label{eqn: A_joint} A &= 
\begin{bmatrix}
  A_1 & 0 & 0 & ... & 0 \\
  0 & A_2 & 0 & ... & 0 \\
  & & ... & &  \\
  0 & ... & 0 & A_{m-1} & 0 \\
  0 & ... & 0 & 0 & A_{m} \\ 
 \end{bmatrix}
\end{align} 


%%%
\mbox{} \newline
\mbox{} \newline
TO FIND M and D: \newline

We define a matrix $D = [D_f \ \ D_p]^T$. To begin, let:
\begin{align*}
D' &= 
\begin{bmatrix}
  X_1 (t_0) \\
  X_1(t_1) \\
  X_2(t_2) \\  
  ... \\
  X_m(t_m) \\
   \dot{X}_1 (t_0) \\
  \dot{X}_1(t_1) \\
  \dot{X}_2(t_2) \\  
  ... \\
  \dot{X}_m(t_m) \\
  \ddot{X}_1 (t_0) \\
   \ddot{X}_1(t_1) \\
 \ddot{X}_2(t_2) \\  
  ... \\
  X^{(r-1)}_m(t_m) \\
 \end{bmatrix}
\end{align*}

We can then rearrange $D'$ into the fixed constraints, $D_f$ and the unfixed constraints, $D_p$. We seek a matrix $M$ such that $d = MD$. 


%%%
% algorithm to find M? 

We can then find:
\begin{align*}
J &= x^T Q x \\
&= (A^{-1} d)^T Q (A^{-1} d) \\
&= d^T A^{-T} Q A^{-1} d \\
&= (MD)^T A^{-T} Q A^{-1} (MD) \\
&= D^T M^T A^{-T} Q A^{-1}  M D
\end{align*}

Define:
\begin{align*}
R &= M^T A^{-T} Q A^{-1}  M \\
&= 
\begin{bmatrix}
R_{FF} & R_{FP} \\
R_{PF} & R_{PP} \\
\end{bmatrix} \\
D &= 
\begin{bmatrix}
D_f \\
D_p \\
\end{bmatrix} \\
D^T &= 
\begin{bmatrix} 
D_f^T & D_p^T \\
\end{bmatrix}
\end{align*}

We can then solve for:
\begin{align*}
J &= 
\begin{bmatrix} 
D_f^T & D_p^T \\
\end{bmatrix}
\begin{bmatrix}
R_{FF} & R_{FP} \\
R_{PF} & R_{PP} \\
\end{bmatrix} 
\begin{bmatrix}
D_F \\
D_P \\
\end{bmatrix} \\
&= D_f^T R_{FF} D_f + D_p^T R_{PF} D_f + D_f^T R_{FP} D_p + D_p^T R_{PP} D_p \\
\frac{dJ}{d D_p} &= D_f^T R_{PF}^T + D_f^T R_{FP} + D_p^T R_{PP} + D_p^T R_{PP}^T 
\end{align*} 

$R$ is symmetric, since:
\begin{align*}
R^T &= (M^T A^{-T} Q A^{-1}  M)^T \\
&= (A^{-1}  M)^T Q^T (M^T A^{-T})^T \\
&= M^T A^{-T} Q A^{-1}  M \\
&= R, \text{ note that Q is symmetric}
\end{align*}

This implies that:
\begin{align*}
R_{PF}^T = R_{FP}, & R_{PP}^T = R_{PP} \\
\frac{dJ}{d D_p} &= 2(D_f^T R_{FP} + D_p^T R_{PP})
\end{align*}

When optimized, 
\begin{align*}
\frac{dJ}{d D_p} &= 2(D_f^T R_{FP} + D_p^T R_{PP}) = 0 \\
D_f^T R_{FP} + D_p^T R_{PP} &= 0 \\
D_p^T &= -D_f^T R_{FP} R_{PP}^{-1} \\
D_p &= -R_{PP}^{-T} R_{FP}^T D_f \\
\end{align*}

Now that we have $D_p$, we can use $D = [D_f \ \ D_p]^T$ to reconstruct the all values for $D'$. This gives us the equality constraint: $A_t X = b_t = D'$.


\mbox{} \newline
We now have the minimization problem:
%%%
\begin{align*}
f(x) &= x^T Q x, \text{subject to: } g(x) = Ax-b = 0
\end{align*}

We can set up an equation using Lagrange multipliers:
\begin{align}
\nonumber \frac{\partial}{\partial x} ( f(x) + \lambda g(x) ) &= 0 \\
\nonumber \frac{\partial}{\partial x} ( x^T Q x + \lambda (Ax-b) ) &= x^T Q + x^T Q^T + \lambda A \\
\nonumber &= 2 x^T Q + \lambda A, \text{ (note that Q is symmetric) } \\
\nonumber &= 0 \\
\label{eqn: analytic} 2Q^T x + A^T \lambda^T &= 0 
\end{align}

Eq. \ref{eqn: analytic} and the constraint $Ax=b$ gives two equations for two unknowns, $x$ and $\lambda$. 
\begin{align*}
\begin{bmatrix}
2Q & A^T \\
A & 0 \\
\end{bmatrix} 
\begin{bmatrix}
x \\
\lambda^T \\
\end{bmatrix} &=
\begin{bmatrix}
0 \\
b \\
\end{bmatrix} \\
\begin{bmatrix}
x \\
\lambda^T \\
\end{bmatrix} &= 
\begin{bmatrix}
2Q & A^T \\
A & 0 \\
\end{bmatrix} ^{-1} 
\begin{bmatrix}
0 \\
b \\
\end{bmatrix}
\end{align*}

$Q$, $A$, and $b$ can be found using Eqs. \ref{eqn: Qkeyframes} and \ref{eqn: endpoint}. 



%\clearpage
%\bibliographystyle{phjcp}
%\bibliography{references}

\end{document}