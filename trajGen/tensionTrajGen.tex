\documentclass[11pt]{article}
\usepackage[margin=1in]{geometry}

\usepackage{graphicx}
\usepackage{amsfonts}
\usepackage{amssymb}
\usepackage{amsmath}
\usepackage{amsthm}
\usepackage{latexsym}
\usepackage{graphicx}
\usepackage{algorithm} 
\usepackage{algorithmic} 
\usepackage{setspace}
\usepackage{subfig}

\begin{document}

\centering
1D Trajectory Generation with Tension Constraints \\
August 22, 2013

\raggedright

\begin{table} [h!]
\footnotesize
\begin{tabular}{ c c }
	$r$ & derivative to minimize in cost function \\
		& implies $r$ initial conditions at each keyframe (position and $r-1$ derivatives, indexed from 0 (constant term) \\
	$n$ & order of desired trajectory, minimum order is $2r-1$ \\
		&  implies $n+1$ coefficients, indexed from 0 (constant term) \\
	$d$ & number of dimensions to optimize, indexed from 1 \\
	 	& for example, $d=3$ when optimizing $[x y z]$ position of a trajectory,  \\
	$m$ & number of pieces in trajectory \\
		& implies $m+1$ keyframes in trajectory, indexed from 1 \\
	$t_{des}$ & vertical vector of desired arrival times at keyframes, indexed from 0 \\
	$p_{des}$ & matrix of desired positions, each row represents a derivative, each column represents a keyframe \\
		&  Inf represents unconstrained 
\end{tabular}
%\caption{Variables used}
\label{tab: vars}
\end{table}







%%%
\newpage
\small


We want to design a 1-dimensional desired trajectory for a load with keyframe positions $p_{L, des} = [X_L(t_0) \ \ X_L(t_1) \ \ X_L(t_2) \ \ ... \ \ X_L(t_m)]^T$ at times $t_{des} = [t_0 \ \ t_1 \ \ t_2 \ \ ... \ \ t_m]^T$ with tensions $T_{des} = [T_0 \ \ T_1 \ \ T_2 \ \ ... \ \ T_m]^T$. For $m$ keyframes, we want a piecewise trajectory $X_L(t)$:
%%
\begin{align*}
X_L(t) &= 
\begin{cases}
    X_{L,1} (t) = c_{1, n} t^n + c_{1, n-1} t^{n-1} + c_{1, n-2} t^{n-2} + ... + c_{1, 0}, & t_0 \le t < t_1 \\
    X_{L,2} (t) = c_{2, n} t^n + c_{2, n-1} t^{n-1} + c_{2, n-2} t^{n-2} + ... + c_{2, 0}, & t_1 \le t < t_2 \\
    ... \\
    X_{L,m} (t) = c_{m, n} t^n + c_{m, n-1} t^{n-1} + c_{m, n-2} t^{n-2} + ... + c_{m, 0}, & t_{m-1} \le t < t_m \\
\end{cases}
\end{align*}

We can solve for this by first finding a piecewise trajectory for the quadrotor, $X(t)$:
%%
\begin{align*}
X(t) &= 
\begin{cases}
    X_{1} (t) = c_{1, n} t^n + c_{1, n-1} t^{n-1} + c_{1, n-2} t^{n-2} + ... + c_{1, 0}, & t_0 \le t < t_1 \\
    X_{2} (t) = c_{2, n} t^n + c_{2, n-1} t^{n-1} + c_{2, n-2} t^{n-2} + ... + c_{2, 0}, & t_1 \le t < t_2 \\
    ... \\
    X_{m} (t) = c_{m, n} t^n + c_{m, n-1} t^{n-1} + c_{m, n-2} t^{n-2} + ... + c_{m, 0}, & t_{m-1} \le t < t_m \\
\end{cases}
\end{align*}

The quadrotor and load states are related by:
\begin{align*}
X(t) &= X_{L}(t)+l \\
X^{(i)}(t) &= X^{(i)}_L (t) 
\end{align*}

For continuity, these relations hold at all keyframes, regardless of $T_{des}$. This implies that the new position constraints for $X(t)$ are $p_{des} = [X(t_0) \ \ X(t_1) \ \ X(t_2) \ \ ... \ \ X(t_m)]^T$ = $p_{L, des} = [X_L(t_0)+l \ \ X_L(t_1)+l \ \ X_L(t_2)+l \ \ ... \ \ X(t_m)+l]^T$. 

\mbox{}\newline
The vector $T_{des}$ is defined such that values are either $\infty$ or $0$, indicating a non-zero or zero tension value, respectively.  For trajectories $X_{L, i}(t)$ where $T_{i-1} > 0$, we want to minimize the 6th derivative of $X_{L, i}(t)$. For trajectory pieces $X_{L, i}(t)$ where $T_{i-1} = 0$, the quadrotor trajectory is independent of the load trajectory and we want to minimize the 4th derivative. 


\mbox{}\newline
Let $x(t)$ be the non-dimensionalized quadrotor trajectory:
%%
\begin{align*}
x(\tau) &= 
\begin{cases}
    x_1 (\tau) = c_{1, n} \tau^n + c_{1, n-1} \tau^{n-1} + ... c_{1, 1} \tau + c_{1, 0}, & t_0 \le t < t_1, \tau = \frac{t-t_0}{t_1-t_0}  \\
    x_2 (\tau) = c_{2, n} \tau^n + c_{2, n-1} \tau^{n-1} + ... c_{2, 1} \tau + c_{2, 0}, & t_1 \le t < t_2, \tau = \frac{t-t_1}{t_2-t_1}  \\
    ... \\
    x_m (\tau) = c_{m, n} \tau^n + c_{m, n-1} \tau^{n-1} + ... c_{m, 1} \tau + c_{m, 0}, & t_{m-1} \le t < t_m, \tau = \frac{t-t_{m-1}}{t_m-t_{m-1}} \\
\end{cases},  0 \le \tau < 1
\end{align*} 
%%
Assume there is only one keyframe $s$ where $T_s = 0$. Let the state vector $x = [x_I \ \ x_{II} \ \ x_{III}]^T$, where $x_I = [c_{1, n} \ \ c_{1, n-1} \ \ ... \ \ c_{s, 0} ]^T$, $x_{II} = [c_{s+1, n} \ \ c_{s+1, n-1} \ \ ... \ \ c_{s+1, 0}]^T$, $x_{III} = [c_{s+2, n} \ \ c_{s+2, n-1} \ \ ... \ \ c_{m, 0} ]^T$, and $x = [c_{1, n} \ \ c_{1, n-1} \ \ ... \ \ c_{m, 1} \ \ ... \ \ c_{m, 0} ]^T$, $x_k = [c_{k, n} \ \ c_{k, n-1} \ \ ... \ \ c_{k, 0} ]^T$. 




%%%
%%%
\newpage
The cost function to minimize is:
%%
\begin{align*}
J &= \int_{t_0}^{t_m} \|  \frac{d^{6} X_I(t) }{dt} \|^2 + \|  \frac{d^{4} X_{II}(t) }{dt} \|^2 + \|  \frac{d^{6} X_{III}(t) }{dt} \|^2 dt \\
&= \sum_{k=1}^{s} \int_{t_{k-1}}^{t_k} \|  \frac{d^{6} X_k (t) }{dt} \|^2 dt +\sum_{k=s+1}^{s+1} \int_{t_{k-1}}^{t_k} \|  \frac{d^{4} X_k (t) }{dt} \|^2 dt +\sum_{k=s+2}^{m} \int_{t_{k-1}}^{t_k} \|  \frac{d^{6} X_k (t) }{dt} \|^2 dt  \\
&= \sum_{k=1}^{s} \int_{0}^{1}  \| \frac{1}{(t_k-t_{k-1})^{6}} \frac{d^{6} x_k (\tau) }{d\tau} \|^2 (t_k-t_{k-1}) d\tau + \\
&  \sum_{k=s+1}^{s+1} \int_{0}^{1}  \|   \frac{1}{(t_k-t_{k-1})^{4}} \frac{ d^{4} x_k (\tau) }{d\tau} \|^2 (t_k-t_{k-1}) d\tau + \sum_{k=s+2}^{m} \int_{0}^{1}  \|  \frac{1}{(t_k-t_{k-1})^{6}} \frac{  d^{6} x_k (\tau) }{d\tau} \|^2 (t_k-t_{k-1}) d\tau  \\
%%
&=  \sum_{k=1}^{s} x_k^T \frac{1}{(t_k-t_{k-1})^{11}} Q_{(0, 1)} x_k + \sum_{k=s+1}^{s+1} x_k^T \frac{1}{(t_k-t_{k-1})^{7}} Q_{(0, 1)} x_k  + \sum_{k=s+2}^{m} x_k^T \frac{1}{(t_k-t_{k-1})^{11}} Q_{(0, 1)} x_k \\
&= x^T Q x 
\end{align*}


%%%
\newpage 
The equality constraints are of the form:
%%
\begin{table}[h!]
\centering
\small
\begin{tabular} {c c c c}
  $t_0, T>0$ & $t_s = t_1, T=0$ & $t_{s+1} = t_2, T>0$ & $t_3,T>0$ \\
  & & & \\
    & & & \\
  $x_1(\tau_0) = X(t_0)$ & $x_1(\tau_1) = X(t_1), x_2(\tau_0) = X(t_1)$ & $x_2(\tau_1) = X(t_2), x_3(\tau_0) = X(t_2)$ & $x_3(\tau_1) = X(t_3) $ \\
    & & & \\
  $\dot{x}_1(\tau_0) = 0$ & $\dot{x}_1(\tau_1) = \dot{x}_2(\tau_0)$ & $\dot{x}_2(\tau_1) =  \dot{x}_3(\tau_0)$ & $\dot{x}_3(\tau_1) = 0 $ \\
    & & & \\
    $\ddot{x}_1(\tau_0) = 0$ & $\ddot{x}_1(\tau_1) = -g, \ddot{x}_2(\tau_0) = -g$ & $\ddot{x}_2(\tau_1) = -g,  \ddot{x}_3(\tau_0) = -g$ & $\ddot{x}_3(\tau_1) = 0 $ \\
      & & & \\
   ${x}^{(3)}_1(\tau_0) = 0$ & ${x}^{(3)}_1(\tau_1) = 0, \ddot{x}^{(3)}_2(\tau_0) = 0$ & ${x}^{(3)}_2(\tau_1) = 0,  x^{(3)}_3(\tau_0) = 0$ & ${x}^{(3)}_3(\tau_1) = 0 $ \\
     & & & \\
   ${x}^{(4)}_1(\tau_0) = 0$ & ${x}^{(4)}_1(\tau_1) = 0, \ddot{x}^{(4)}_2(\tau_0) = 0$ & ${x}^{(4)}_2(\tau_1) = 0,  x^{(4)}_3(\tau_0) = 0$ & ${x}^{(4)}_3(\tau_1) = 0 $ \\
     & & & \\
   ${x}^{(5)}_1(\tau_0) = 0$ & ${x}^{(5)}_1(\tau_1) = 0, \ddot{x}^{(5)}_2(\tau_0) = 0$ & ${x}^{(5)}_2(\tau_1) = 0,  x^{(5)}_3(\tau_0) = 0$ & ${x}^{(5)}_3(\tau_1) = 0 $ \\
%     & & & \\
%   ${x}^{(6)}_1(\tau_0) = 0$ & ${x}^{(6)}_1(\tau_1) = 0, \ddot{x}^{(6)}_2(\tau_0) = 0$ & ${x}^{(6)}_2(\tau_1) = 0,  x^{(6)}_3(\tau_0) = 0$ & ${x}^{(6)}_3(\tau_1) = 0 $ \\
     & & & \\
     \hline
      6 const. & 11 const. & 11 const. & 6 const.\\
\end{tabular}
\end{table}

\mbox{} \newline
Equation for finding $X(t_{s+1})$:
\[
\boxed{
X(t_{s+1}) - \dot{X}(t_s)(t_{s+1}-t_s) - X(t_s) = -\frac{g}{2}(t_{s+1}-t_s)^2
}
\]

Or, in the non-dimensionalized case, $x_{s+1}(\tau_1) = x_{s+2}(\tau_0)$:
\[
\boxed{
x_{s+1}(\tau_1) - \dot{x}_{s+1}(\tau_0) - x_{s+1}(\tau_0)  = -\frac{g}{2} (t_{s+1}-t_s)^2
}
\]


%%%
\newpage
The inequality constraints are of the form (where $t_c$ is the time of a sample point and $\tau_c = \frac{t_c - t_i}{t_f-t_i}$)

\begin{align*}
T > 0 & \text{ on } x_I \text{ and } x_{III} \\
-T &< 0 \\
-(m_L (\ddot{X}_L(t)+g) &\le -\epsilon \\
-(m_L (\ddot{X}(t)+g) &\le -\epsilon \\
\end{align*}

\[
\boxed{
-\ddot{x}(t_c) \le  (-\frac{ \epsilon }{m_L} + g ) (t_f-t_i)^2 
}
\]

\begin{align*}
 X(t) - X_L(t) > 0 & \text{ on } x_{II} \\
 X_L(t) -X(t) &\le - \epsilon \\
 - X(t) & \le - \epsilon -  X_L(t) \\
\end{align*}

\[
\boxed{
- x(\tau_c) +  \dot{x} (\tau_0) \tau_c +  x(\tau_0) \le - \epsilon + \frac{g}{2} \tau_c^2 (t_f-t_i)^2 + l
}
\]

\begin{align*}
  X(t) - X_L(t) < l  & \text{ on } x_{II} \\
 X(t) - X_L(t) &\le l - \epsilon \\
  X(t) & \le l - \epsilon +  X_L(t) \\
\end{align*}

\[
\boxed{
x(\tau_c) -  \dot{x} (\tau_0) \tau_c -  x(\tau_0) \le - \epsilon -\frac{g}{2} \tau_c^2 (t_f-t_i)^2
}
\]

\begin{align*}
   \dot{X}_L(t_{s+1}) - \dot{X}(t_{s+1}) & \le 0 \\
   \dot{X}_L(t_{s+1}) & \le  \dot{X}(t_{s+1})  \\
\end{align*}

\[
\boxed{
\dot{x}_{s+1}(\tau_0) - \dot{x}_{s+1}(\tau_1) \le  g(t_{s+1}-t_s)^2
}
\]








%%%
%%%
\newpage
TO FIND Q: \newline
When $x' = [c_0 \ \ c_1 \ \ ... \ \ c_{n-1} \ \ c_n]^T$, we can find $Q'_{(0, 1)}$ with: 
%%
\begin{align}
\label{eqn: Q}  Q'[i, j]_{(t_0, t_1)} &= 
\begin{cases}
    \prod_{k = 0}^{r-1} {(i-k)(j-k)} \frac{ t_1^{i+j-2r+1} - t_0^{i_j-2r+1} }{i+j-2r+1}, & i \ge r \land j \ge r \\
    0, & i < r \lor j < r 
\end{cases}, i = 0...n, j = 0...n
\end{align}
%%
Since our $x_k = [c_{k, n} \ \ c_{k, n-1} \ \ ... \ \ c_{k, 1} \ \ c_{k, 0}]^T$, reflecting $Q'$ horizontally and vertically will give us the desired $Q$ for the form of $x_k$. We can then create the block diagonal matrix:
\begin{align}
\label{eqn: Qkeyframes} Q &= 
\begin{bmatrix}
  \frac{1}{(t_1-t_{0})^{2r-1}} Q_{(0, 1)} & 0 & 0 & ... & 0 \\
  0 & \frac{1}{(t_2-t_{1})^{2r-1}} Q_{(0, 1)} & 0 & ... & 0 \\
  & & ... & &  \\
  0 & ... & 0 & \frac{1}{(t_{m-1}-t_{m-2})^{2r-1}} Q_{(0, 1)} & 0 \\
  0 & ... & 0 & 0 & \frac{1}{(t_m-t_{m-1})^{2r-1}} Q_{(0, 1)} \\ 
 \end{bmatrix},
\end{align}
%%
where $r = 6$ if $k \ne s+1$ and $r = 4$ if $k=s+1$. 



%%%
\mbox{} \newline
\mbox{} \newline
TO FIND $A_{EQ}$: \newline

\mbox{} \newline
%%%
At keyframes $0$ and $m$, we have a full set of constraints for position and derivatives 1-5. At $t_s$ and $t_{s+1}$, where $T_{s} = 0$, $\ddot{X} = -g$, $X^{(3)} = X^{(4)} = X^{(5)} = 0$. At all other points, the keyframe positions are constrained.While $X(t_i)$ is specified for $i \ne s+1$, $X(t_{s+1})$ is calculated from free fall conditions of the load. The equation for free fall of the load is:
\begin{align*}
%\[ 
%\boxed{ 
X_L(t_{s+1}) = -\frac{g}{2}(t_{s+1}-t_s)^2+\dot{X}_L(t_s)(t_{s+1}-t_s) + X_L(t_s) 
%} 
%\]
\end{align*}
%%
Translating this to the quadrotor equation: 
\begin{align*}
X(t_{s+1}) &= X_L(t_{s+1}) + l \\
&= -\frac{g}{2}(t_{s+1}-t_s)^2+\dot{X}_L(t_s)(t_{s+1}-t_s) + X_L(t_s) + l \\
\text{From $X = X_L + l$ and $\dot{X} = \dot{X}_L$ }& \\
&= -\frac{g}{2}(t_{s+1}-t_s)^2+\dot{X}(t_s)(t_{s+1}-t_s) + X(t_s)
\end{align*}
%%
Nondimensionalizing with:
\begin{align*}
%\tau &= \frac{t-t_i}{t_f-t_i}, t = \tau(t_f-t_i)+t_i \\
%t_{s+1}-t_s &= \tau_1(t_{s+1}-t_s)+t_i - ( \tau_0(t_{s+1}-t_s)+t_i ) \\
%&= (\tau_1 - \tau_0)(t_{s+1}-t_s) \\
%X(t_{s+1}) = x_{s+1}(\tau_1)&= -\frac{g}{2}(\tau_1 - \tau_0)^2 (t_{s+1}-t_s)^2+\dot{X}(t_s)(\tau_1 - \tau_0)(t_{s+1}-t_s) + X(t_s) \\
%&= -\frac{g}{2}(\tau_1 - \tau_0)^2 (t_{s+1}-t_s)^2+\frac{ \dot{x}_{s+1}(\tau_0) } { t_{s+1}-t_s } (\tau_1 - \tau_0)(t_{s+1}-t_s) + x_{s+1}(\tau_0) \\
%&=  -\frac{g}{2}(\tau_1 - \tau_0)^2 (t_{s+1}-t_s)^2+ \dot{x}_{s+1}(\tau_0)  (\tau_1 - \tau_0)+ x_{s+1}(\tau_0) \\
X(t_{s+1}) = x_{s+1}(\tau_1)&= -\frac{g}{2} (t_{s+1}-t_s)^2+\dot{X}(t_s)(t_{s+1}-t_s) + X(t_s) \\
&= -\frac{g}{2} (t_{s+1}-t_s)^2+\frac{ \dot{x}_{s+1}(\tau_0) } { t_{s+1}-t_s } (t_{s+1}-t_s) + x_{s+1}(\tau_0) \\
&=  -\frac{g}{2} (t_{s+1}-t_s)^2+ \dot{x}_{s+1}(\tau_0) + x_{s+1}(\tau_0) \\
\end{align*}

At keyframe $t_{s+1}$ when the quad and load collide, there is a change in load velocity from the collision. Assuming a completely inelastic collision: 
%%
\begin{align*}
(\dot{X}^+ - \dot{X}_L^+) &= e ( \dot{X}^- + \dot{X}_L^-)\\
m_L \dot{X}_L^+ + m \dot{X}^+ &= m_L \dot{X}_L^- + m \dot{X}^- \\
\text{For $e = 0$, } & \\
\dot{X}^+ &= \dot{X}_L^+ \\
m_L \dot{X}_L^- + m \dot{X}^- &= (m_L+m) \dot{X}^+ \\
\text{If $m >> m_L$, } (m_L+m) \approx m, m_L \approx 0 & \\
 \dot{X}^+ &= \dot{X}^-
\end{align*}
%%
We want the quadrotor velocity at $t_{s+1}$ to be continuous and can be chosen independently of the load velocity. 
%We want the quadrotor velocity at $t_{s+1}$ to be continuous and also match the load velocity at that time for a smooth transition. At the end of free-fall, the load velocity is: 
%\begin{align*}
%\dot{X}_L(t_{s+1}) &= -g (t_{s+1}-t_s)+\dot{X}_L(t_s) \\
%\frac{ \dot{x}_{L, s+1} (\tau_1) }{t_{s+1}-t_s} &= - g (t_{s+1}-t_s)+\frac{ \dot{x}_{L, s+1}(\tau_0) }{ t_{s+1}-t_s } \\
% \dot{x}_{L, s+1}(\tau_1)  &= - g (t_{s+1}-t_s)^2 +  \dot{x}_{L, s+1}(\tau_0) \\
% &= - g (t_{s+1}-t_s)^2 +  \dot{x}_{s+1}(\tau_0) \\
% \text{We thus want to set } & \\
%\dot{x}_{s+1}(\tau_1) = \dot{x}_{s+2}(\tau_0) &=   \dot{x}_{L, s+1} (\tau_1) = - g (t_{s+1}-t_s)^2 +  \dot{x}_{L, s+1} (\tau_0) \\
%\dot{x}_{s+1} (\tau_1) - \dot{x}_{s+1} (\tau_0) &=    - g (t_{s+1}-t_s)^2  \\
%\dot{x}_{s+2}(\tau_0) - \dot{x}_{s+1} (\tau_0) &=  - g (t_{s+1}-t_s)^2  \\
%\end{align*}


\mbox{} \newline
There $26+2m$ constraints - 6 from each endpoint, $2(m-1)$ additional position constraints, $(4)(4)$ constraints to account for constraints on derivatives 2-5 at $t_{s}$ and $t_{s+1}$.  They are: 
%%%
\begin{align}
\nonumber A_{endpoint} x &= b_{endpoint} \\
\label{eqn: endpoint} \begin{bmatrix}
%constraints for xi
x_1(\tau_0) \\
\dot{x}_1 (\tau_0) \\
\ddot{x}_1 (\tau_0) \\
x^{(3)}_1 (\tau_0) \\
x^{(4)}_1 (\tau_0) \\
x^{(5)}_1 (\tau_0) \\
x_1 (\tau_1) \\
x_2 (\tau_0) \\
x_2 (\tau_1) \\
x_3 (\tau_0) \\ 
... \\
x_{s} (\tau_0) \\
x_{s} (\tau_1) \\
\ddot{x}_{s} (\tau_1) \\
x^{(3)}_{s} (\tau_1) \\
x^{(4)}_{s} (\tau_1) \\
x^{(5)}_{s} (\tau_1) \\
%%
%%%
% constraints for xii
x_{s+1} (\tau_0) \\
\ddot{x}_{s+1} (\tau_0) \\
x^{(3)}_{s+1} (\tau_0) \\
x^{(4)}_{s+1} (\tau_0) \\
x^{(5)}_{s+1} (\tau_0) \\
x_{s+1} (\tau_1) - \dot{x}_{s+1}{\tau_0}(\tau_1-\tau_0) - x_{s+1}(\tau_0) \\
\ddot{x}_{s+1} (\tau_1) \\
x^{(3)}_{s+1} (\tau_1) \\
x^{(4)}_{s+1} (\tau_1) \\
x^{(5)}_{s+1} (\tau_1) \\
%%%
%%%
% constraints for xiii
x_{s+2} (\tau_0) - \dot{x}_{s+1}{\tau_0}(\tau_1-\tau_0) - x_{s+1}(\tau_0)\\
\ddot{x}_{s+2} (\tau_0) \\
x^{(3)}_{s+2} (\tau_0) \\
x^{(4)}_{s+2} (\tau_0) \\
x^{(5)}_{s+2} (\tau_0)  \\
x_{s+2} (\tau_1) \\
x_{s+3} (\tau_0) \\
... \\
x_{m} (\tau_0) \\
x_{m} (\tau_1) \\
\dot{x}_{m} (\tau_1) \\
\ddot{x}_{m} (\tau_1) \\
x^{(3)}_{m} (\tau_1) \\
x^{(4)}_{m} (\tau_1) \\
x^{(5)}_{m} (\tau_1) \\
 \end{bmatrix}
 &= 
 \begin{bmatrix}
 % xi
  X (t_0) \\
  (t_1-t_0) \dot{X} (t_0) \\
  (t_1-t_0)^{(2)}  X^{(2)} (t_0) \\
    (t_1-t_0)^{(3)}  X^{(3)} (t_0) \\
      (t_1-t_0)^{(4)}  X^{(4)} (t_0) \\
  (t_1-t_0)^{(5)}  X^{(5)} (t_0) \\
    X (t_1) \\
     X (t_1) \\
     X (t_2) \\
X (t_2) \\
  ... \\
  X(t_{s-1}) \\
       X (t_s) \\
 -g (t_s-t_{s-1})^{(2)}  \\
 0 \\
 0 \\
 0 \\
  %%
 %%
 %xii
         X (t_{s}) \\
 -g (t_{s+1}-t_{s})^{(2)}  \\
 0 \\
 0 \\
 0 \\
  -\frac{g}{2} (\tau_1-\tau_0)^2 (t_{s+1}-t_s)^2 \\        %X (t_{s+1}) \\
 -g (t_{s+1}-t_{s})^{(2)}  \\
 0 \\
 0 \\
 0 \\
   %%
 %%
 % xiii
  -\frac{g}{2} (\tau_1-\tau_0)^2 (t_{s+1}-t_s)^2 \\  %          X (t_{s+1}) \\
 -g (t_{s+2}-t_{s-1})^{(2)}  \\
 0 \\
 0 \\
 0 \\
X(t_{s+2}) \\
X(t_{s+2}) \\
... \\
    X(t_{m-1}) \\
    X(t_m) \\
  (t_m-t_{m-1}) \dot{X} (t_m) \\
  (t_m-t_{m-1})^{(2)}  X^{(2)} (t_m) \\
     (t_m-t_{m-1})^{(3)}  X^{(3)} (t_m) \\
    (t_m-t_{m-1})^{(4)}  X^{(4)} (t_m) \\
   (t_m-t_{m-1})^{(5)}  X^{(5)} (t_m) \\ 
 \end{bmatrix} 
\end{align}


%%
We also need continuity of derivatives lower than 6 at all intermediate keyframes, which gives an additional $5(m-3)+2$ constraints - $5(m-3)$ continuity constraints at each keyframe $i$ such that $i \ne 0, m, s s+1$ and 2 for velocity continuity at $t_{s}$ and $t_{s+1}$. Explicitly: 
 \begin{align*}
 A_{cont} x &= b_{cont} \\
 \begin{bmatrix}
 %% xi continuity 
  \frac{1}{(t_1-t_0)} \dot{x}_1 (\tau_1) - \frac{1}{(t_2-t_1)} \dot{x}_2(\tau_0) \\
  ... \\
  \frac{1}{(t_1-t_0)^{(5)}} x^{(5)}_1 (\tau_1) - \frac{1}{(t_2-t_1)^{(5)}} x^{(5)}_2 (\tau_0) \\
  ... \\
  \frac{1}{(t_{s-1}-t_{s-2})^{(5)}} x^{(5)}_{s-1} (\tau_1) - \frac{1}{(t_s-t_{s-1})^{(5)}} x^{(5)}_s (\tau_0) \\
  %%%
  %%%
  %xii velocity constraints
 \frac{1}{(t_{s}-t_{s-1})} \dot{x}_{s} (\tau_1) -  \frac{1}{(t_{s+1}-t_{s})}  \dot{x}_{s+1} (\tau_0) \\
  \frac{1}{(t_{s+1}-t_{s})} \dot{x}_{s+1} (\tau_1) -  \frac{1}{(t_{s+2}-t_{s+1})}  \dot{x}_{s+2} (\tau_0) \\
  %%%
  %%%
  % xiii
  \frac{1}{(t_{s+2}-t_{s+1})} \dot{x}_{s+2} (\tau_1) - \frac{1}{(t_{s+3}-t_{s+2})} \dot{x}_{s+3}(\tau_0) \\
  ... \\
  \frac{1}{(t_{s+2}-t_{s+1})^{(5)}} x^{(5)}_{s+2]} (\tau_1) - \frac{1}{(t_{s+3}-t_{s+2})^{(5)}} x^{(5)}_{s+3} (\tau_0) \\
  ... \\
  \frac{1}{(t_{m-1}-t_{m-2})^{(5)}} x^{(5)}_{m-1} (\tau_1) - \frac{1}{(t_m-t_{m-1})^{(5)}} x^{(5)}_m (\tau_0) \\
 \end{bmatrix} 
 &= 0
 \end{align*}
 
 
 
 
The full set of $7m+13$ constraints, $Ax = b$, are:
\begin{align}
\nonumber Ax &= b \\
\label{eqn: Akeyframes} 
\begin{bmatrix}
A_{endpoint} \\
A_{cont} \\
 \end{bmatrix}
 x 
& = 
 \begin{bmatrix}
b_{endpoint} \\
0 \\
 \end{bmatrix} 
\end{align}

Note that for $m$ pieces, we have $12m$ coefficients. Thus, we are potentially missing $5m-13$ constraints for a fully constrained system.




%%%
\mbox{} \newline
\mbox{} \newline
TO FIND $A_{INEQ}$:
 
We need 3 sets of inequality constraints. In general, we choose $N_c$ intermediate points between $t_f$ and $t_i$, located at times $t_c = t_i+\frac{k}{N_c+1}$ for $k = 1, 2, ..., N_c$ and evaluate the inequality constraints at these points. Note that the inequality constraints are not evaluated at trajectory endpoints. 

\mbox{} \newline
When the rope is taut, tension at $t_c$ is always greater than 0, or $T_i(t_c) > 0$ for all $i \ne s+1$:
\begin{align*}
T &> 0 \\
-T &< 0 \\
-(m_L (\ddot{X}_L(t)+g) &\le -\epsilon \\
-(m_L (\frac{ \ddot{x}(t_c) }{(t_f - t_i)^2}+g) &\le -\epsilon \\
-\ddot{x}(t_c) &\le  (-\frac{ \epsilon }{m_L} + g ) (t_f-t_i)^2 \\
\end{align*}
 
 
 \mbox{} \newline
During this free-fall phase, the position of the load $X_L(t_c)$ can be explicitly calculated as a function of time and the quadrotor state at $X(t_s)$:
  \begin{align*}
X_L(t_c) &= -\frac{g}{2} (t_c-t_i)^2 + \dot{X}_L (t_i) (t_c-t_i) + X_L(t_i) \\
&= -\frac{g}{2} (t_c-t_i)^2 + \dot{X} (t_i) (t_c-t_i) + X(t_i) - l \\
\tau = \frac{t-t_i}{t_f-t_i}, & t = \tau (t_f-t_i) + t_i \\
t_c &= t_i + \frac{p}{N_c+1} (t_f-t_i) \\
&= (\tau_0 (t_f-t_i) + t_i) + \frac{p}{N_c+1} \left( (\tau_1 (t_f-t_i) + t_i) - (\tau_0 (t_f-t_i) + t_i) \right) \\
&= \tau_0 (t_f-t_i) + t_i + \frac{p}{N_c+1} (t_f-t_i) (\tau_1-\tau_0) \\
t_c - t_i &= \tau_0 (t_f-t_i) + \frac{p}{N_c+1} (t_f-t_i) (\tau_1-\tau_0) \\
X_L(\tau_c) &=  -\frac{g}{2} (t_c-t_i)^2 + \dot{X} (t_i) (t_c-t_i) + X(t_i) - l \\
&=  -\frac{g}{2} ( \tau_0 (t_f-t_i) + \frac{p}{N_c+1} (t_f-t_i) (\tau_1-\tau_0) )^2 + \frac{ \dot{x} (\tau_0) }{t_f-t_i} ( \tau_0 (t_f-t_i) + \frac{p}{N_c+1} (t_f-t_i) (\tau_1-\tau_0) ) + x(\tau_0) - l \\
&=  -\frac{g}{2} \tau_c^2 (t_f-t_i)^2 + \dot{x} (\tau_0) \tau_c + x(\tau_0) - l \\
 \end{align*}
 
  When the rope is slack, the quad and load shouldn't collide with each other, giving the constraint:
   \begin{align*}
 X(t_c) - X_L(t_c) &> 0 \\
 X_L(t_c) -X(t_c) &\le - \epsilon \\
 - X(t_c) & \le - \epsilon -  X_L(t_c) \\
  & \le - \epsilon + \frac{g}{2} \tau_c^2 (t_f-t_i)^2 - \dot{x} (\tau_0) \tau_c - x(\tau_0) + l \\
  - x(\tau_c) +  \dot{x} (\tau_0) \tau_c +  x(\tau_0) & \le - \epsilon + \frac{g}{2} \tau_c^2 (t_f-t_i)^2 + l
 \end{align*}
 
  In addition, the rope should remain slack:
 \begin{align*}
 X(t_c) - X_L(t_c) &< l \\
 X(t_c) - X_L(t_c) &\le l - \epsilon \\
  X(t_c) & \le l - \epsilon +  X_L(t_c) \\
  & \le l - \epsilon -\frac{g}{2} \tau_c^2 (t_f-t_i)^2 + \dot{x} (\tau_0) \tau_c + x(\tau_0) - l \\
  x(\tau_c) -  \dot{x} (\tau_0) \tau_c -  x(\tau_0) & \le - \epsilon -\frac{g}{2} \tau_c^2 (t_f-t_i)^2
 \end{align*}
 
Finally, at the moment before the rope becomes taut again, we have the constraint to ensure that the load is in fact falling away from the quad: 
 \begin{align*}
 \dot{X}_L(t_{s+1}) - \dot{X}(t_{s+1}) & \le 0 \\
  \dot{X}_L(t_{s+1}) - \frac{\dot{x}_{s+1}(\tau_1)}{(t_{s+1}-t_s)} & \le 0 \\
 \dot{X}_L(t_{s+1}) &= -g(t_{s+1}-t_s) + \dot{X}_L(t_s) \\
&= -g(t_{s+1}-t_s) +  \frac{\dot{x}_{L, s+1}(\tau_0)}{(t_{s+1}-t_s)} \\
&= -g(t_{s+1}-t_s) +  \frac{\dot{x}_{s+1}(\tau_0)}{(t_{s+1}-t_s)} \\
 -g(t_{s+1}-t_s) +  \frac{\dot{x}_{s+1}(\tau_0)}{(t_{s+1}-t_s)} - \frac{\dot{x}_{s+1}(\tau_1)}{(t_{s+1}-t_s)} & \le 0 \\
 \dot{x}_{s+1}(\tau_0) - \dot{x}_{s+1}(\tau_1) & \le  g(t_{s+1}-t_s)^2
 \end{align*}

\end{document}